\section{Technologieübergreifende Schwachstellen}

Vorweg die Frage, sind das alles deine Themen? Oder wolltest du das aufteilen?

\begin{itemize}
   \item Race Conditions
   \item Sichere Kommunikationswege
   \item Kryptografie
\end{itemize}

Was willst du hier schreiben? Über Kryptografie habe ich einen alleine einen vierstündigen Vortrag gehalten.


\subsection{Unterscheidung in Technologien}

\begin{itemize}
   \item Webanwendungen
   \item "FatClient"-Anwendungen
\end{itemize}

Hier würde ich das anders machen, eher auf den Punkt OpenSource und kommerzieller Software eingehen. Denn darun liegen auch wehsentliche Untershiede. Bezüglich API, Schnittstellen usw.


\subsection{Typische webbasierte Schwachstellen}
\begin{itemize}
      \item Cross-Site Scripting\\
        Erläuterung von XSS\\ 
      	\textbf{Gegenmaßnahme:} Ein- und Ausgabevalidierung
      \item Cross-Site-Request Forgery\\
        Erläuterung von CSRF\\ 
      	\textbf{Gegenmaßnahme:} Serverseitig ausgestelltes CSRF Token
      \item SQL Injection\\
        Erläuterung von SQL-Injection\\ 
      	\textbf{Gegenmaßnahme:} Eingabevalidierung 
      \item Clientseitige Sicherheitsmechanismen\\
        Erläuterung von clientseitig implementierten Sicherheitsmechanismen z.B. im Kontext von JS  
        \textbf{Gegenmaßnahme:} Serverseitige Validierung von Eingabeparametern
      \item Session-Management
        Erläuterung von Schwachstellen im Session Management einer Webapplikation\\ 
      	\textbf{Gegenmaßnahme:} Cookie Flags verwenden, sinnvolle Session Timeouts, etc. 
      \item Optional: Technische Gegenmaßnahmen, Einführung einer Webapplication-Firewall oder einen OS Produkt wie "mod\_security"
      \item Optional: Erläuterung von "Security"-Frameworks im Webumfeld wie z.B. Apache Wicket
      \item Optional: Was bringt HTML5 für neue, möglicherweise sicherheitskritische Features mit sich?
      \begin{itemize}
         \item Cross Origin Ressource Sharing: Erleichtert XSS ("Shell of the Future", "Beef"-Framework) 
         \item Support von SVG-Grafiken: Einbetten von JS Code in Grafiken die innerhalb der Applikation nachgeladen werden
         \item Webstorage: Speicherung von sensitiven Daten innerhalb des Webstorage des Browsers, Zugriff per XSS auf den Webstorage
      \end{itemize}      
\end{itemize}


\subsection{Typische "FatClient"-Schwachstellen}

\begin{itemize}
      \item BufferOverflow\\
        Erläuterung von BufferOverflows\\ 
      	\textbf{Gegenmaßnahme:} Verwendung von sicheren Funktionen
      \item Als weitere Gegenmaßnahme für Buffer Overflows die Funktion von ASLR erläutern
      \item Verwendung von proprietärer Kryptografie\\
        Erläuterung von Schwierigkeiten durch den Verwendung von proprietärer Kryptografie\\ 
      	\textbf{Gegenmaßnahme:}  Auf offene Algorithmen (Standards) zurückgreifen
      \item Optional: Querverweis auf Microsofts SDL
      \item Optional einen Querverweis auf Tools die im Nachgang potentiell unsichere Anwendungen absichern z.B. MS EMET
\end{itemize}


\subsection{Technologieübergreifende Lösungsansätze}

\begin{itemize}
      \item Regelmäßige Durchführung von Penetrationstest
      \item Einführung von Standards, im Kontext von Anwendungssicherheit vielleicht einen Querverweis auf ISO 27001. Natürliche keine vollständige Einführung eines ISMS sonder eher den PDCA-Zyklus betrachten
      \item Einführung von Bugtracking-Systemen
      \item Durchführung von Awareness-Maßnahmen 
      \item Einführung von Entwicklungsrichtlinien 
\end{itemize}

Ich verstehe noch nicht warum ein Bugtracking-System die Software sicherer und robuster macht ;-) Vielleicht könntest du das mal ein wenig mehr ausführen, was du dir hier vorstellst.
