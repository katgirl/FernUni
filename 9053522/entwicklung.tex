\section{Softwareentwicklung}

Die Softwareentwicklung selbst ist so alt wie das Computerzeitalter. 
Mit den ersten Computern wurden auch die ersten Programme entwickelt,
diese waren noch lange nicht so umfangreich wie die heutigen, so wurden
auch schon in den frühen Anfängen einfache mathematische Operationen durchgeführt. 

Mit der immer leistungsfähiger werdenden Hardware, werden Softwareentwicklungen
die einst als sicher galten heute als unsicher eingestuft. Alleine immer
bessere Hardware schafft mehr Opperationen in kürzerer Zeit zu berrechnen.
Was zu Folge hat, dass auch komplexe Algorithmen in immer kürzerer Zeit
gelöst werden. Was eigentlich als technischer Fortschritt und Verbesserung 
angesehen werden kann, führt aber auch zu neuen Problemen.
Um vertrauliche Informationen unleserlich zu machen, reichte es Cäsar 
ein Stück Papyrus um einen zylindrischen Holzstab zu wickeln und die 
Informationen so unleserlich zu machen.\footnote{http://de.wikipedia.org/wiki/Caesar-Verschlüsselung} 
Im zweiten Weltkrieg wurde dann die Enigma \footnote{http://de.wikipedia.org/wiki/Enigma_(Maschine)} 
lange Zeit eingesetzt um eine sichere Kommunikation zu gewährleisten. 
Erst in den 1970er Jahren gab es einen Wechsel von symmetrische auf 
asymmetrische Verschlüsselung, wobei Sender und Empfänger nicht mehr auf 
den gleichen Schlüssel angewiesen waren. Seid dem wird der als sicher
geltende Schlüssel gefühlt jährlich verdoppelt. Zudem gesellen sich immer
neuere Verschlüsselungsalgorithmen, die auf mathematischen Grundlagen basieren. 
So wird die 1993 entwickelte Blowfish-Verschlüsselung\footnote{http://www.schneier.com/blowfish.html}
heute bevorzugt verwendet und ersetzt damit den SHA-Algorythmus\footnote{http://csrc.nist.gov/publications/fips/fips180-2/fips180-2withchangenotice.pdf}. 
In PHP5.5 wurde Blowfish als erste Passwortverschlüsselung in die neue
Hash-Funktion aufgenommen.  

Es findet ein ständigen Wettkampf mit immer besser werdenden
Hardware und komplexeren Softwareanwendungen statt. So kann konnte man 
2000 noch ein Programm nach den damaligen "Sicherheitsrichtlininen" 
entwickeln, welches heute als risikobehaftet gilt. So entstehen 
neue Herausforderungen die es zu meistern gilt. 
Softwareentwicklungen müssen oft in einem zeitlich vorgeschriebenen 
Rahmen fertiggestellt werden. Daher wird oft geschaut welche Lebensdauer 
die Software hat und für welchen Einsatzzweck sie bestimmt ist. Bei einigen
Projekten wird dann ein Risikomanagement durchgeführt, das den 
Schaden schon vor der eigentlichen Entwicklung abschätzen soll.

\subsection{Programmiersprachen}
Im einundzwanzigsten Jahrhundert bewegt man sich in einem multikulturellen 
Umfeld, in mehrerlei Hinsicht.

So gab es schon in den Anfängen des Computerzeitalters gleich dutzende 
von Programmiersprachen, die aber jeweils speziell an die Bedürfnisse der 
Anforderung angepasst waren. Mit Basic und C entstanden die ersten universellen
Programmiersprachen die die Entwicklung rasch voran trieben und heute 
noch eingesetzt werden.

Dann kam in den 1990er-Jahren das Internet, welches neue Anforderungen
an die Programmiersprachen stellte und durch die rasche Entwicklung auch 
viele Programmiersprachen sterben ließ, da sie mit dem Fortschritt 
nicht mithalten konnten.

Das war die Geburtsstunde der objektorientierten und skriptbasierten 
Programmiersprachen.

Durch die Globalisierung und damit schnellere Kommunikationswege und 
einer einheitlichen Sprache (englisch), kann man sich auch über große 
Distanzen austauschen und Wissen teilen.

Das heute verbreitetste Programmierparadigma ist das Objektorientierte.
Dabei wird versucht, die Daten als Objekt zu betrachten und so die 
Komplexität zu vereinfachen, indem Redundanzen nutzbar gemacht werden.
Auch heute gibt es noch zahlreiche Programmiersprachen, die ihre Anhänger
haben, die sich in sogennante Communitys zusammenschließen um die 
Programmiersprachen weiter zu entwickeln und das Wissen auszutauschen.

\subsection{Frameworks} 
Ein Framework stellt kein fertiges Programm dar, vielmehr dient es 
Entwicklerinnen und Entwicklern als Grundgerüst in der 
objektorientierten Programmierung. Es bringt dabei für gewöhnlich neben 
der Struktur auch die Anwendungsarchitektur mit. Insbesondere wird 
aber der Kontrollfluss der Anwendung und die Schnittstellen darüber
in architektonischen Mustern bereitgestellt. 
Heute gibt es zahlreiche Frameworks zu fast jeder Programmiersprachen und
Anwendungsfall, daher gibt es keine allgemeingültige Definition.   

\minisec{Framework Typen}
\begin{itemize}
  \item Application Frameworks 
        \newline bilden das Programmiergerüst für Anwendungen.
  \item Domain Frameworks 
        \newline bilden das Programmiergerüst für Problembereich.
  \item Class Frameworks 
        \newline bilden über Klassen und Methoden die Abstraktionsebenen 
        für ein breites Anwendungsfeld ab.
  \item Komponenten-Frameworks
        \newline bilden die Basis für Software-Komponenten, indem sie die 
        objektorientierte Ebenen abstrahieren.
  \item Coordination-Frameworks
        \newline bilden die Basis für Geräte-Interaktion.
  \item Tests Frameworks
        \newline dienen als Programmiergerüst für (automatisierte) Softwaretests.
  \item Webframeworks
        \newline sind speziell für Webanwendungen ausgelegt.
\end{itemize}

\subsection{Entwicklungsumgebungen}
Eine Entwicklungsumgebungen oder kurz IDE (Integrierte Entwicklungsumgebung)
ist eine Sammlung verschiedener Anwendungsprogramme. Somit kann man 
Medienbrüche in der Softwareentwicklung vermeiden. Was gerade bei größeren
Entwicklungen zu Effizienz in der Arbeit führt.
Es gibt für fast jede Programmiersprache eine große Vielzahl von 
Entwicklungsumgebungen. Sowohl Open-Source als auch proprietäre 
Entwicklungsumgebungen.
So ist die wohl bekannteste und verbreitetste Entwicklungsumgebung für
Softwareentwicklungen die in und ursprünglich für Java geschriebene 
Umgebung Eclipse\footnote{http://www.eclipse.org/}. Die Entwicklungsumgebung
erfreute sich nicht zuletzt durch den 2001 von IBM freigebenden Quellcode
großer Beliebtheit. Auch die Weiterentwicklung ist zur Zeit gesichert, 
da IBM weiterhin Entwickler für die Arbeit an der Software finanziert. 
Durch Erweiterungen erden weitere Programmiersprachen
von Eclipse unterstützt. Dadurch ist jeder der mit Softwareentwicklung
zu tun hat schon mal mit Eclipse in Berührung gekommen.  
Dieses ist aber lange nicht die einzige Erfolgsgeschichte auf dem Markt
der Entwicklungsumgebungen. So gibt es zahlreiche Produkte, die um die 
Gunst der Softwareentwickler buhlen. 

\subsection{Versionskontrollsystem}
Bei einem Versionskontrollsystem werden Änderungen dokumentiert und
festgehalten. So lassen sich Veränderungen an einem Dokument auch
über viele folgende Änderungen nachschlagen und rückgängig machen bzw.
nachverfolgen warum eine Änderung vorgenommen wurde.
Es gibt verschiedene Funktionsweisen:
\begin{itemize}
  \item Lokale Versionsverwaltung
  \item Zentrale Versionsverwaltung 
  \item Verteilte Versionsverwaltung 
\end{itemize}
Die wohl bekanntesten Vertreter solcher Software sind Subversion (SVN)
\footnote{http://subversion.apache.org} und git\footnote{http://git-scm.com}, die
zudem noch unter freien Lizenzen stehen.
Dabei zählt SVN zu den zentralen Versionsverwaltung und git zur 
verteilten Versionsverwaltung.
SVN ist wohl die bekannteste Versionsverwaltung und mit Version 1.1 schafft
sie bereits 2004 den Durchbruch, als Änderungen nicht mehr in der Datenbank, 
sondern im Dateisystem abgelegt werden konnten.
git ist erst 2005 durch Linus Torvalds entwickelt worden, nachdem die
Lizenz für das bisher von ihm eingesetzte Versionskontrollsystem geändert 
wurde. Der große Durchbruch gelang aber erst durch github, einer Plattform
auf der man seine Projekte teilen und mit anderen bearbeiten kann.

\subsection{Ticketsystem}
In einem Ticketsystem werden Fehler oder aber neue Funktionswünsche 
durch Mitglieder/Kunden/... gemeldet, die dann von einem Entwickler 
oder einer Entwicklerin betrachtet, nachvollzogen und klassifiziert
werden können. 
Voraussetzung für eine Fehlerbehebung ist eine genaue Fehlerbeschreibung 
und ein reproduzierbarer Weg um den Fehler zu erzeugen/hervorzurufen.
Durch verschiedene Statuse kann dann dem Nutzer mitgeteilt werden wann 
oder in welcher Version der Fehler behoben ist.
In Open-Souce-Projekten sind solche Ticketsystem zudem offen, so das 
andere die Problemstellung einsehen können und ggf. einen besseren
oder anderen Lösungsansatz vorab diskutieren können.

\subsection{Softwarearten}
Mit Softwarearten ist in diesem Kapitel die Art der Entstehung/Entwicklung
der Software gemeint.
Weithin gibt es zwei Arten der Softwarentwicklung:
\begin{itemize}
  \item kommerzielle Softwareentwicklung
  \item freie Softwareentwicklung
\end{itemize}
Mit "frei" ist in diesem Fall Software gemeint, wo der Quellcode der
Software frei von jedem Nutzer eingesehen werden kann. Sehr häufig wird
"frei" mit open-source gleichgesetzt, was aber defakto nicht der Fall ist.
\minisec{kommerzielle Softwareentwicklung} 
Bei kommerzieller Softwareentwicklung handelt es sich häufig um 
Auftragsprojekte durch einen Kunden bzw. Eigenentwicklungen um ein
Softwareprodukt auf dem Markt zu platzieren.
Bei kommerziellen Softwareentwicklungen kennt oft nur ein erlesener Kreis
an Personen den Quellcode und ist auch für dessen Qualität verantwortlich.
Zwar wird hier versucht, hochwertige Arbeit abzuliefern, doch leider
ist das nicht immer der Fall. Teilweise wird der Code an externe
Firmen zur Sicherheitsprüfung rausgegeben, da die Gewinnoptimierung 
stets eine große Rolle spielt, wird leider aber genauso oft auf solche Tests
verzichtet und darauf vertraut, das die eigenen Entwickler/Entwicklerinnen
den Code geprüft haben und an alle Sicherheitsmerkmale gedacht haben.
\minisec{freie Softwareentwicklung}   
Bei quellcodefreier Software kann jeder vorher selber prüfen, wenn er dazu
selbst in der Lage ist, oder jemanden beauftragen die Software zu prüfen.
In Open-Source Communitys passiert dieses häufig automatisch. Da hier 
mehrere Entwickler/Entwicklerinnen an einem Projekt arbeiten wird ein
Versionskontrollsystem eingesetzt. Hier kann jede Veränderung nachverfolgt 
werden. Durch die Verknüpfung mit einem Ticketsystem sind Nutzer in der 
Lage, Wünsche und Fehler zu melden, die dann behoben oder
aufgenommen werden können. Anschließend schauen mehrere unabhängige 
Entwickler/Entwicklerinnen über den neu hinzugefügten Code und geben
ihr Urteil darüber ab. 
Durch diese Art der Entwicklung findet gleichzeitig eine Prüfung der
Qualität und Sicherheit der Software statt.  


\subsection{Softwareentwicklung durch \textit{Security by Design}}
Der Frauenhofer Verlag hat auch in diesem Jahr eine aktualisierte 
Ausgabe seines Trend- und Strategiebericht zu "Entwicklung sicherer 
Software durch Security by Design"
\footnote{https://www.sit.fraunhofer.de/fileadmin/dokumente/studien_und_technical_reports/Trendbericht_Security_by_Design.pdf} 
herausgegeben.
Dabei geht es darum, bei der Softwarenetwicklung bestimmte
Sicherheitsaspekte wie Lebenszyklus und die Integration von 
Softwarekomponenten anderer Hersteller frühzeitig zu analysieren.
Dieser Trend- und Strategiebericht soll nicht als Richtlinie verstanden
werden, eher soll er als Leitfaden dienen um sich mit der Problemstellung
zu befassen.
Das 65 Seitige Werk geht dabei auf die Bedeutung von Security By Design ein
und zeigt wie durch Automatisierung und Reduktion menschlicher Fehlereinflüsse
Code sicherer gemacht werden kann. Weiterhin bietet es einen Einblick
in Probleme die Legacy-Software mit sich bringt und zeigt an verteilter 
Entwicklung und Interagtion das kaum noch ein Softwareprojekt von einem
einzigen Entwicklerteam realisiert wird und wo hier die Gefahren liegen.
Den Abschluss macht ein Blick in die Zukunft, wo es darum geht, das
schon heute einen Schritt in die richtige Richtung gegangen ist, 
aber noch einen weiten Weg bevorsteht bis es keine Meldungen
über Sicherheitslücken und Angriffe mehr gibt.
