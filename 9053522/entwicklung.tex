\section{Softwareentwicklung}\label{entwicklung}

Die Softwareentwicklung selbst ist so alt wie das Computerzeitalter. 
Mit den ersten Computern wurden auch die ersten Programme entwickelt,
diese waren noch lange nicht so umfangreich wie die heutigen, doch wurden
auch schon in den frühen Anfängen einfache mathematische Operationen durchgeführt. 

Mit der immer leistungsfähiger werdenden Hardware, werden Softwareentwicklungen
die einst als sicher galten heute als unsicher eingestuft. Alleine immer
bessere Hardware schafft mehr Opperationen in kürzerer Zeit zu berrechnen.
Was zu Folge hat, dass auch komplexe Algorithmen in immer kürzerer Zeit
gelöst werden.  
Um vertrauliche Informationen unleserlich zu machen, reichte es Cäsar, 
im alten Ägypten, ein Stück Papyrus um einen Holzstab zu wickeln und
die Informationen so unleserlich zu machen. 
\footnote[1]{http://de.wikipedia.org/wiki/Caesar-Verschlüsselung} 
Im zweiten Weltkrieg wurde dann die Enigma \footnote[2]{http://de.wikipedia.org/wiki/Enigma_(Maschine)} 
lange Zeit eingesetzt um eine sichere Kommunikation zu gewährleisten. 
Erst in den 1970er Jahren gab es einen Wechsel von symmetrischen auf 
asymmetrische Verschlüsselung, wobei Sender und Empfänger nicht mehr auf 
den Gleichen Schlüssel angewiesen waren. Seid dem wird der als sicher
geltende Schlüssel gefühlt jährlich verdoppelt. Zudem gesellen sich immer
neuere Verschlüsselungsalgorithmen, die auf mathematischen Grundlagen basieren. 

Wenn man gerne sichere Software erstellen möchte als Entwickler, so 
befindet man sich in einem ständigen Wettkampf mit immer besser werdender 
Hardware und komplexeren Softwareanwendungen. Hierdurch entstehen neue 
Herausforderungen die es zu meistern gilt. 
Zudem müssen Softwareentwicklungen oft in einem zeitlich vorgeschriebenen 
Rahmen fertiggestellt werden. Daher wird oft geschaut welche Lebensdauer 
die Software hat und für welchen Einsatzzweck sie bestimmt ist. Zu diesem 
Zweck gibt es ein Risikomanagement, das den Schaden schon vor der eigentlichen
Entwicklung abschätzen soll.

\subsubsection{Programmiersprachen}
Wir leben heute in einem multikulturellen Umfeld, in mehrerlei Hinsicht.
So gab es schon in den Anfängen des Computerzeitalters gleich dutzende 
von Programmiersprachen, die aber speziell an die Bedürfnisse der Anforderung
angepasst waren. Mit Basic und C entstanden die ersten universellen
Programmiersprachen die die Entwicklung rasch voran trieben und heute 
noch eingesetzt werden. 

Neben den Programmiersprachen entstanden auch Programmierparadigmen.
Die wohl heute verbreitetste Programmierparadigma ist die opjektorientierte.
Dabei wird versucht die Daten als Objekt zu betrachten und so die 
Komplexität zu vereinfachen indem Redundanzen nutzbar gemacht werden.

 

  

 
