\fancyhead[L]{} %Kopfzeile links
\section{Softwareentwicklung}

Die Softwareentwicklung selbst ist so alt wie das Computerzeitalter. 
Mit den ersten Computern wurden auch die ersten Programme entwickelt,
diese waren noch lange nicht so umfangreich wie die heutigen, jedoch wurden
auch schon in den frühen Anfängen einfache mathematische Operationen durchgeführt. 

Aufgrund der immer leistungsfähiger werdenden Hardware werden Softwareentwicklungen, 
die einst als sicher galten heute als unsicher eingestuft. Allein die stetig 
verbesserte Hardware schafft es, mehr Operationen in kürzerer Zeit zu berechnen.
Was zur Folge hat, dass auch komplexe Algorithmen in immer kürzerer Zeit
gelöst werden. Was eigentlich als technischer Fortschritt und Verbesserung 
angesehen werden kann, führt aber auch zu neuen Problemen.
Um vertrauliche Informationen unleserlich zu machen, reichte es Julius Cäsar, 
ein Stück Papyrus um einen zylindrischen Holzstab \cite{caesar_verschlusselung} zu wickeln. %\footnote{\url{http://de.wikipedia.org/wiki/Caesar-Verschlüsselung}} 
Im zweiten Weltkrieg wurde die Chiffriermaschine Enigma \cite{enigma_maschine} %\footnote{\url{http://de.wikipedia.org/wiki/Enigma_(Maschine)}} 
lange Zeit eingesetzt, um eine sichere Kommunikation zu gewährleisten. 
Erst in den 1970er-Jahren gab es einen Wechsel von symmetrischer auf 
asymmetrische Verschlüsselung, wobei Sender und Empfänger nicht mehr auf 
den gleichen Schlüssel angewiesen waren. Seitdem wird der als sicher
geltende Schlüssel gefühlt jährlich verdoppelt. Dazu kommen immer
neuere Verschlüsselungsalgorithmen, die auf mathematischen Grundlagen basieren. 
So wird die 1993 entwickelte Blowfish-Verschlüsselung \cite{schneier_blowfish} %\footnote{\url{http://www.schneier.com/blowfish.html}}
heute bevorzugt verwendet und ersetzt damit den SHA-Algorithmus \cite{secure_hash_standard} %\footnote{\url{http://csrc.nist.gov/publications/fips/fips180-2/fips180-2withchangenotice.pdf}}. 
In PHP5.5 wurde Blowfish als erste Passwortverschlüsselung in die neue
Hash-Funktion aufgenommen.   

Es findet ein ständiger Wettkampf im Hardware-Bereich statt und auch die 
immer komplexer werdenden Software-Anwendungen konkurrieren. Ein Programm, 
das im Jahr 2000 nach den damaligen "Sicherheitsrichtlininen" 
entwickelt wurde, gilt heute meist als risikobehaftet. So entstehen 
neue Herausforderungen, die es zu meistern gilt. 
Softwareentwicklungen müssen oft in einem zeitlich vorgeschriebenen 
Rahmen fertiggestellt werden. Daher wird oft geschaut, welche Lebensdauer 
die Software hat und für welchen Einsatzzweck sie bestimmt ist. Bei einigen
Projekten wird dann ein Risikomanagement durchgeführt, das den 
Schaden schon vor der eigentlichen Entwicklung abschätzen soll.

\newpage
\fancyhead[L]{\nouppercase{\leftmark}} %Kopfzeile links
\subsection{Programmiersprachen}
Im 21. Jahrhundert bewegt man sich, in mehrerlei Hinsicht, in einem multikulturellen 
Umfeld. So gab es schon in den Anfängen des Computerzeitalters gleich Dutzende 
von Programmiersprachen, die aber jeweils an spezielle Bedürfnisse angepasst 
waren. Mit Basic und C entstanden die ersten universellen
Programmiersprachen, die die Entwicklung rasch vorantrieben und heute 
noch eingesetzt werden. Dann kam in den 1990er-Jahren das Internet, welches 
neue Anforderungen an die Programmiersprachen stellte und durch die rasche 
Entwicklung auch viele Programmiersprachen sterben ließ, da sie mit dem 
Fortschritt nicht mithalten konnten. Das war die Geburtsstunde der 
objektorientierten und skriptbasierten Programmiersprachen. Durch die 
Globalisierung und die damit einhergehenden schnelleren Kommunikationswege 
sowie die Verwendung von Englisch als Weltsprache kann man sich heute auch über große 
Distanzen austauschen und Wissen teilen. Das aktuell verbreitetste Programmierparadigma 
ist das Objektorientierte. Dabei wird versucht, die Daten als Objekt zu betrachten und so die 
Komplexität zu vereinfachen, indem Redundanzen nutzbar gemacht werden.

Auch heute gibt es noch zahlreiche Programmiersprachen, die ihre Anhänger
haben, die sich in sogenannte Communitys zusammenschließen, um die 
Programmiersprachen weiter zu entwickeln und das Wissen auszutauschen.

\subsection{Frameworks} 
Ein Framework stellt kein fertiges Programm dar, vielmehr dient es 
Entwicklerinnen und Entwicklern als Grundgerüst in der 
objektorientierten Programmierung. Es bringt dabei für gewöhnlich neben 
der Struktur auch die Anwendungsarchitektur mit. Insbesondere wird 
aber der Kontrollfluss der Anwendung und die Schnittstellen darüber
in architektonischen Mustern bereitgestellt. 
Heute gibt es zahlreiche Frameworks zu fast jeder Programmiersprache und
jedem Anwendungsfall, daher gibt es keine allgemeingültige Definition.   

\newpage
\minisec{Framework Typen}
\begin{itemize}
  \item Application Frameworks 
        \newline bilden das Programmiergerüst für Anwendungen.
  \item Domain Frameworks 
        \newline bilden das Programmiergerüst für Problembereich.
  \item Class Frameworks 
        \newline bilden über Klassen und Methoden die Abstraktionsebenen 
        für ein breites Anwendungsfeld ab.
  \item Komponenten-Frameworks
        \newline bilden die Basis für Software-Komponenten, indem sie die 
        objektorientierte Ebenen abstrahieren.
  \item Coordination-Frameworks
        \newline bilden die Basis für Geräte-Interaktion.
  \item Tests Frameworks
        \newline dienen als Programmiergerüst für (automatisierte) Softwaretests.
  \item Webframeworks
        \newline sind speziell für Webanwendungen ausgelegt.
\end{itemize}

\subsection{Entwicklungsumgebungen}
Eine Entwicklungsumgebung oder kurz IDE (Integrierte Entwicklungsumgebung)
ist eine Sammlung verschiedener Anwendungsprogramme. Damit kann man 
Medienbrüche in der Softwareentwicklung vermeiden, was gerade bei größeren
Entwicklungen zu Effizienz in der Arbeit führt.
Es gibt für fast jede Programmiersprache eine große Vielzahl von 
Entwicklungsumgebungen. Sowohl Open-Source als auch proprietäre 
Entwicklungsumgebungen.
So ist die wohl bekannteste und verbreitetste Entwicklungsumgebung für
Softwareentwicklungen die in und ursprünglich für Java geschriebene 
Umgebung Eclipse\footnote{\url{http://www.eclipse.org}}. Die Entwicklungsumgebung
erfreute sich nicht zuletzt durch den 2001 von IBM freigebenden Quellcode
großer Beliebtheit. Auch die Weiterentwicklung ist zur Zeit gesichert, 
da IBM weiterhin Entwickler für die Arbeit an der Software finanziert. 
Durch Erweiterungen werden weitere Programmiersprachen
von Eclipse unterstützt. Dadurch ist jeder, der mit Softwareentwicklung
zu tun hat schon mal mit Eclipse in Berührung gekommen.  
Dieses ist aber lange nicht die einzige Erfolgsgeschichte auf dem Markt
der Entwicklungsumgebungen. So gibt es zahlreiche Produkte, die um die 
Gunst der Softwareentwickler buhlen. 

\subsection{Versionskontrollsystem}
Bei einem Versionskontrollsystem werden Änderungen dokumentiert und
festgehalten. So lassen sich Veränderungen an einem Dokument auch
über viele folgende Änderungen nachschlagen und rückgängig machen bzw.
nachverfolgen, warum eine Änderung vorgenommen wurde.
Es gibt verschiedene Funktionsweisen:
\begin{itemize}
  \item Lokale Versionsverwaltung
  \item Zentrale Versionsverwaltung 
  \item Verteilte Versionsverwaltung. 
\end{itemize}
Die wohl bekanntesten Vertreter solcher Software sind Subversion (SVN)\footnote{\url{http://subversion.apache.org}} und git\footnote{\url{http://git-scm.com}}, die
zudem noch unter freien Lizenzen stehen.
Dabei zählt SVN zu den zentralen Versionsverwaltung und git zur 
verteilten Versionsverwaltung.
SVN ist wohl die bekannteste Versionsverwaltung und mit Version 1.1 schaffte
sie bereits 2004 den Durchbruch, als Änderungen nicht mehr in der Datenbank, 
sondern im Dateisystem abgelegt werden konnten.
git ist erst 2005 durch Linus Torvalds entwickelt worden, nachdem die
Lizenz für das bisher von ihm eingesetzte Versionskontrollsystem geändert 
wurde. Der große Durchbruch gelang aber erst durch github, einer Plattform
auf der man seine Projekte teilen und mit anderen bearbeiten kann.

\subsection{Ticketsystem}
In einem Ticketsystem werden Fehler oder aber neue Funktionswünsche 
durch Mitglieder/Kunden/... gemeldet, die dann von einem Entwickler 
oder einer Entwicklerin betrachtet, nachvollzogen und klassifiziert
werden können. 
Voraussetzung für eine Fehlerbehebung ist eine genaue Fehlerbeschreibung 
und ein reproduzierbarer Weg, um den Fehler zu erzeugen/hervorzurufen.
Durch verschiedene Status kann dann dem Nutzer mitgeteilt werden, wann 
oder in welcher Version der Fehler behoben ist.
In Open-Source-Projekten sind solche Ticketsysteme zudem offen, sodass 
andere die Problemstellung einsehen können und ggf. einen besseren
oder anderen Lösungsansatz vorab diskutieren können.

\newpage
\subsection{Softwarearten}
Mit Softwarearten ist in diesem Kapitel die Art der Entstehung/Entwicklung
der Software gemeint.
Weithin gibt es zwei Arten der Softwarentwicklung:
\begin{itemize}
  \item kommerzielle Softwareentwicklung
  \item freie Softwareentwicklung.
\end{itemize}
Mit "frei" ist in diesem Fall Software gemeint, wo der Quellcode der
Software frei von jedem Nutzer eingesehen werden kann. Sehr häufig wird
"frei" mit Open Source gleichgesetzt, was aber de facto nicht der Fall ist.
\minisec{Kommerzielle Softwareentwicklung} 
Bei kommerzieller Softwareentwicklung handelt es sich häufig um 
Auftragsprojekte durch einen Kunden bzw. Eigenentwicklungen, um ein
Softwareprodukt auf dem Markt zu platzieren.
Bei kommerziellen Softwareentwicklungen kennt oft nur ein erlesener Kreis
an Personen den Quellcode und ist auch für dessen Qualität verantwortlich.
Zwar wird hier versucht, hochwertige Arbeit abzuliefern, doch leider
ist das nicht immer der Fall. Teilweise wird der Code an externe
Firmen zur Sicherheitsprüfung herausgegeben, da die Gewinnoptimierung 
stets eine große Rolle spielt, wird leider aber genauso oft auf solche Tests
verzichtet und darauf vertraut, dass die eigenen Entwickler/Entwicklerinnen
den Code geprüft haben und an alle Sicherheitsmerkmale gedacht haben.
\minisec{Freie Softwareentwicklung}   
Quellcodefreie Software kann jeder selbst prüfen - vorausgesetzt, er/sie verfügt über 
das erforderliche Wissen - oder aber jemanden mit der Prüfung beauftragen.
In Open-Source-Communitys passiert dies häufig automatisch. Da hier 
mehrere Entwickler/Entwicklerinnen an einem Projekt arbeiten, wird ein
Versionskontrollsystem eingesetzt. Hier kann jede Veränderung nachverfolgt 
werden. Durch die Verknüpfung mit einem Ticketsystem sind Nutzer in der 
Lage, Wünsche und Fehler zu melden, die dann behoben oder
aufgenommen werden können. Anschließend schauen mehrere unabhängige 
Entwickler/Entwicklerinnen über den neu hinzugefügten Code und geben
ihr Urteil darüber ab. 
Durch diese Art der Entwicklung findet gleichzeitig eine Prüfung der
Qualität und Sicherheit der Software statt.  

\newpage
\subsection{Softwareentwicklung durch \textit{Security by Design}}
Der Frauenhofer Verlag hat auch in diesem Jahr eine aktualisierte 
Ausgabe seines Trend- und Strategieberichts zur "Entwicklung sicherer 
Software durch Security by Design" %\cite{waidner_hrsg_trendbericht_2013} %\footnote{\url{https://www.sit.fraunhofer.de/fileadmin/dokumente/studien_und_technical_reports/Trendbericht_Security_by_Design.pdf}} 
herausgegeben.
Dabei geht es darum, bei der Softwareentwicklung bestimmte
Sicherheitsaspekte wie Lebenszyklus und die Integration von 
Softwarekomponenten anderer Hersteller frühzeitig zu analysieren.
Dieser Trend- und Strategiebericht soll nicht als Richtlinie verstanden
werden, eher soll er als Leitfaden dienen, um sich mit der Problemstellung
zu befassen.
Das 65-seitige Werk geht dabei auf die Bedeutung von "Security By Design" ein
und zeigt, wie durch Automatisierung und Reduktion menschlicher Fehler
Code sicherer gemacht werden kann. Weiterhin bietet es einen Einblick
in Probleme, die Legacy-Software mit sich bringt und zeigt an verteilter 
Entwicklung und Interaktion, dass kaum noch ein Softwareprojekt von einem
einzigen Entwicklerteam realisiert wird und wo hier die Gefahren liegen.
Den Abschluss bildet ein Blick in die Zukunft: Man sei schon einen Schritt in die richtige Richtung gegangen, doch es  
stünde noch ein weiter Weg bevor, bis es keine Meldungen mehr 
über Sicherheitslücken und Angriffe gäbe.
