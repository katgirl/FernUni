\section{Umgang mit Sicherheitslücken}
Wie geht man mit dem Bekanntwerden von Sicherheitslücken in Software um?

Bisher wurde beschrieben, welche Arten von Sicherheitslücken es gibt und 
wie man Code aktuell wirkungsvoll davor schützt. Desweiteren wurden aktuelle 
Lösungswege gezeigt, um sie zu vermeiden.

In den meisten Fällen kann man bei Bekanntwerden von Sicherheitslücken
diese beseitigen und ein Update bereitstellen.
Was aber passiert mit Code, der eine Sicherheitslücke hat, die
kritisch ist, aber nicht zeitnah aktualisiert werden kann?

Aktuell gibt es zahlreiche solcher bekannten Sicherheitslücken. Der wohl 
populärste Hacker im Dienste des "Guten" war der Sicherheitsexperte 
Barnaby Jack, der zu den sogenannten \textit{white hats} zählte. Er starb 
vergangene Woche an bislang ungeklärter Ursache im Alter von 35 Jahren
%\footnote{\url{http://www.spiegel.de/netzwelt/web/hacker-barnaby-jack-in-san-francisco-gestorben-a-913380.html}}. 
\cite{spon_barnaby_jack}.
Kommende Woche wollte er auf einer Konferenz über die Sicherheit von 
Herzschrittmachern sprechen.
In den Medien sorgt aktuell noch ein weiterer Fall für Schlagzeilen.
Dabei geht es um geknackte Wegfahrsperrcodes \cite{spon_vw} %\footnote{\url{http://www.spiegel.de/auto/aktuell/volkswagen-erwirkt-verfuegung-gegen-akademische-codeknacker-a-913462.html}} 
bei den Luxusmarken der Volkswagen Gruppe (VW).
Hier wurde von VW kurzfristig eine Verfügung veranlasst, die den Wissenschaftlern
eine Veröffentlichung untersagt.

Beide Beispielfälle zeigen Sicherheitslücken in Systemen, die
nur schwer aktualisiert werden können. 
Die Softwareentwicklung bei einem Herzschrittmacher benötigt viele
Tests und einen Langzeittest um sicherzustellen, dass durch ein Update
nicht eine Fehlfunktion zum Herzstillstand des Patienten führt. 
Bei VW muss jetzt jeder betroffene Wagen in eine Werkstatt, um mit neuer
Software ausgestattet zu werden, da man nicht einfach das Update über 
eine Datenleitung einspielen kann. 
Zwar gibt es immer mehr Systeme, die eine Anbindung an das Internet haben \cite{internet_der_dinge}
%\footnote{\url{http://www.n-tv.de/ratgeber/Sendungen/Wenn-nicht-nur-der-Kuehlschrank-online-ist-article10431986.html}}, 
doch bringt dieses nicht nur Vorteile, da gerade die globale Erreichbarkeit
neue Sicherheitsprobleme hervorruft.  
Für Hersteller sind nicht nur die Fehlerbehebung und die Tests mit 
Kosten verbunden, sondern auch eine eventuelle Rufschädigung. So versuchen
gerade große Konzerne, die Hacker ihrer Systeme als Sicherheitsexperten 
zu gewinnen.

