\section{Einleitung}\label{einleitung}


% Vorschlag, wir lassen uns auf nichts festnagelen sondern halten die Einleitung generisch :)
% Bitte meinen Kram nur als Vorschlag sehen! 


%In kommerziellen Softwareprojekten arbeitet ein geschlossener Kreis an
%Personen an dem Code. Dieser wird dann in der Regel vor Veröffentlichung durch 
%entsprechende Experten geprüft und freigegeben.
%\\
%Bei OpenSource Projekten, wo der der Quellcode frei zugänglich ist und eine Mitarbeit 
%auch gewünscht ist, kann hingegen jeder der programmieren kann oder auch denkt den Code
%erweitern und eigene Module entwickeln.
%\\
%Bei vielen Modulen wird dabei oft die API ignoriert oder es schleichen sich schnell 
%kritische Sicherheitlücken ein.
%\\
%In der nachfolgenden Hausarbeit geht es um die Sicherheit in der Programmierung. 
%Dabei soll auf typische Fehler eingegangen werden und anhand von unterschiedlichen 
%Angriffsszenarien, Techniken und Technologien gezeigt werden wo die häufigsten 
%Schwachstellen liegen und wie man diese ganz einfach beseitigen kann.
%Ausgangspunkt ist die Modulare Objekt orientierter Programmierung, 
%die heute in fast allen Softwareprojekten zur Anwendung kommt.\\

% ToDo: In der Einleitung die Vielschichtigkeit der ganzen Thematik betonen, 
% sichere Programmierug hängt von vielen Faktoren ab, z.B.
%
% Qualität des Codereviews u.U. belegen, dass Opensource-Entwicklungen besser/schlecher/gleich im Vergleich mit 
% mit "komerziellen" Entwicklungen abschneiden
%
% Den Begriff "kommerziell" im Kontext dieser Arbeit klar abgrenzen

Jede Entwicklerin und jeder Entwickler möchte neben der Funktionalität 
auch die Sicherheit ihrer/seiner Software garantieren.

Um die Aspekte der sicheren Programmierung zu betrachten, 
muss man sich im Vorfeld über die grundlegenden Risiken bei der 
Entwicklung von Softwarelösungen informieren.
Im Kontext der Informationstechnik haben sich drei primäre Schwachpunkte
herauskristallisiert:

\begin{itemize}
      \item\textbf{Vertraulichkeit}\\
 	   Eine Softwarelösung führt eine ausreichende Überprüfung von 
 	   Benutzerberechtigungen durch und erlaubt nur legitimen Benutzern 
 	   den Zugriff auf (vertrauliche) Daten. 
 	   Die Verdaulichkeit von Daten hat implizit rechtliche Auswirkungen 
 	   und ist in vielen Ländern durch entsprechende nationale Gesetze 
 	   geregelt (z.B. Bundesdatenschutzgesetz). Neben rechtlichen 
 	   Auswirkungen hat die Vertraulichkeit von Daten im Hinblick auf 
 	   wirtschaftliche Interessen (z.B. Diebstahl von Forschungsergebnissen) 
 	   einen sehr hohen Stellenwert.
	  
	  \item\textbf{Integrität}\\
	   Eine Softwarelösung verhindert eine Manipulation von Daten oder 
	   stellt Methoden (z.B. Hashwertbildung) zu Feststellung einer 
	   unerlaubten Modifikation der Daten bereit.
	  
	  \item\textbf{Verfügbarkeit} 
	   Eine Softwarelösung muss funktionieren, auch wenn ein Fehler- 
	   oder Problemfall auftritt. Die Verfügbarkeit ist aus wirtschaftlicher 
	   Sicht analog zur Vertraulichkeit zu bewerten. Fällt in einem 
	   Unternehmen eine zentrale Systemkomponente (z.B. SAP-System) 
	   aufgrund eines Softwarefehlers aus, kann dies zum vollständigen 
	   erliegen eines Geschäftsprozesses führen und in kürzester Zeit 
	   einen immensen monetären Schaden verursachen.
\end{itemize}

Da zur Wahrung der oben aufgeführten Schutzziele im Umfeld der 
Softwareentwicklung vielfältige Lösungsansätze existieren, würde eine 
detaillierte Betrachtungen der verschiedenen Entwicklungsparadigmen, 
Implementierungsverfahren und Frameworks den Rahmen dieser Ausarbeitung 
bei Weitem überschreiten. Aus diesem Grund betrachten die Verfasser 
ausgewählte Entwicklungsprozesse und Konzepte, die eine sichere 
Entwicklung von Software unterstützen. Daneben werden verschiedene 
Angriffs- und Bedrohungsszenarien einschließlich möglicher 
Gegenmaßnahmen aufgezeigt.


