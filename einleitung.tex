\section{Einleitung}\label{einleitung}

In kommerziellen Softwareprojekten arbeitet ein geschlossener Kreis an
Personen an dem Code. Dieser wird dann in der Regel vor Veröffentlichung durch 
entsprechende Experten geprüft und freigegeben.
\\
Bei OpenSource Projekten, wo der der Quellcode frei zugänglich ist und eine Mitarbeit 
auch gewünscht ist, kann hingegen jeder der programmieren kann oder auch denkt den Code
erweitern und eigene Module entwickeln.
\\
Bei vielen Modulen wird dabei oft die API ignoriert oder es schleichen sich schnell 
kritische Sicherheitlücken ein.
\\
In der nachfolgenden Hausarbeit geht es um die Sicherheit in der Programmierung. 
Dabei soll auf typische Fehler eingegangen werden und anhand von unterschiedlichen 
Angriffsszenarien, Techniken und Technologien gezeigt werden wo die häufigsten 
Schwachstellen liegen und wie man diese ganz einfach beseitigen kann.
Ausgangspunkt ist die Modulare Objekt orientierter Programmierung, 
die heute in fast allen Softwareprojekten zur Anwendung kommt.\\
