\section{Fazit}
Durch die globalisierte Vernetzung nimmt die Sicherheit in der 
Softwareentwicklung einen immer größeren Stellenwert ein. In den 
vergangenen Jahren ist die Komplexität mit jeder neu entstandenen 
Software gestiegen. 
\newline

%~ \textit{„Wann haben Sie zuletzt Ihren Smart-Fernseher aktualisiert?“} 
%~ \begin{flushright}
%~ Marc Rogers, US-Sicherheitsforscher
%~ \end{flushright} 
%~ \newline

Obwohl Sicherheitslücken wie SQL-Injection und Cross-Site Scripting (XSS), 
wie in dieser Hausarbeit beschrieben, schon seit Jahren bekannt sind und 
effektiv bekämpft werden können, liest man noch heute immer wieder über diese 
Schwachstellen. Inzwischen gibt es eigene Firmen, die bezahlt werden, um 
Sicherheitslücken zu finden und damit das Softwareprodukt sicherer zu 
machen. Doch durch die stetige Weiterentwicklung und neuen Möglichkeiten
entstehen wiederum neue Sicherheitslücken.
Auch wenn eine Schwachstelle behoben wurde bedeutet dieses nicht gleich, 
dass alle betroffenen Systeme von der Behebung profitieren.
Ein bekanntes Beispiel sind Content Managment Systeme, die von
Administratoren oft nur schlecht gewartet werden. So gibt es Schätzungen,
das heute nur ca. 50\% aller Wordpress-Installationen \footnote{http://playground.ebiene.de/wordpress-versionen-verteilung/}
aktuell sind. Sicherheitsexperten gehen sogar aufgrund der stetigen
Entwicklung von Hardware und Softwarekomplexitäten davon aus, dass 
nach einem Jahr ohne Aktualisierungen eine Software hochgradig
anfällig für Schwachstellen ist.
Auch der gerne benutzte Punkt "Sicherheit durch Unbekanntheit" 
kann nicht genutzt werden. Gerade in Zeiten von Spionage untersuchen
Hacker Webseiten auf Auffälligkeiten.
 
Daher müssen sich nicht nur Entwickler mit der ständigen Weiterentwicklung 
auseinandersetzen, sondern auch die Nutzer dieser Software selbst. 
Auch wenn gleich es inzwischen viele Hilfsprogramme gibt, die nachschauen 
ob es eine neuere Version gibt, so muss schlussendlich doch auch der 
Nutzer tätig werden, um das Update durchzuführen.   
